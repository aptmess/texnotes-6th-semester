\RequirePackage{ifluatex}
\let\ifluatex\relax

\documentclass[aps,%
12pt,%
final,%
oneside,
onecolumn,%
musixtex, %
superscriptaddress,%
centertags]{article} %% 
\topmargin=-40pt
\textheight=650pt
\usepackage[english,russian]{babel}
\usepackage[utf8]{inputenc}
%всякие настройки по желанию%
\usepackage[colorlinks=true,linkcolor=black,unicode=true]{hyperref}
\usepackage{euscript}
\usepackage{supertabular}
\usepackage[pdftex]{graphicx}
\usepackage{amsthm,amssymb, amsmath}
\usepackage{textcomp}
\usepackage[noend]{algorithmic}
\usepackage[ruled]{algorithm}
\usepackage{lipsum}
\usepackage{indentfirst}
\usepackage{babel}
\usepackage{pgfplots}
\usepackage{setspace}
\linespread{1.15}
\pgfplotsset{compat=1.9}

\pgfplotsset{model/.style = {blue, samples = 100}}
\pgfplotsset{experiment/.style = {red}}

\selectlanguage{russian}

\setlength{\parindent}{2.4em}
\setlength{\parskip}{0.1em}
%\renewcommand{\baselinestretch}{2.0}

\usepackage{xcolor}
\usepackage{hyperref}
 
 % Цвета для гиперссылок
%\definecolor{linkcolor}{HTML}{799B03} % цвет ссылок
%\definecolor{urlcolor}{HTML}{799B03} % цвет гиперссылок
 
%\hypersetup{pdfstartview=FitH,  linkcolor=linkcolor,urlcolor=urlcolor, colorlinks=true}

\begin{document}

\begin{titlepage} 
\begin{center}
% Upper part of the page
%\textbf{\Large САНКТ-ПЕТЕРБУРГСКИЙ ГОСУДАРСТВЕННЫЙ ЭКОНОМИЧЕСКИЙ УНИВЕРСИТЕТ} \\[1.0cm]
%\textbf{\large Кафедра Прикладной Математики и Информатики}\\[3.5cm]
 
% Title
\textbf{}\\[10.0cm]
\textbf{\LARGE Теория игр}\\[0.5cm]
\textbf{\Large ПМ-1701} \\[0.1cm]

%supervisor
\begin{center} \large
{Преподаватель:} \\[0.5cm]
\textsc {Чернов Виктор Петрович}\\
{viktor\_chernov@mail.ru}\\
\end{center}
% \begin{flushright} \large
%\emph{Рецензент:} \\
%д.ф. - м.н., профессор \textsc{Надеемся Нам Помогут}
%\end{flushright}
%\begin{flushright} \large
%\emph{Заведующий кафедрой:} \\
%д.ф. - м.н., профессор \textsc{Не Обмани Себя}
%\end{flushright}
\vfill 

% Bottom of the page
{\large {Санкт-Петербург}} \par
{\large {2020 г., 6 семестр}}
\end{center} 
\end{titlepage}

% Table of contents
\begin{thebibliography}{3}
\bibitem{eliseeva}
Теория игр
\end{thebibliography}
\tableofcontents
\newpage

\section{10.02.2020}
\subsection{Введение}

Предметом теории игр является моделированием конфликтных ситуаций. Зачинателем "Теории игр" является Джон фон Нейман, а последователем является Джон Нэш. Зададимся вопросом, а как описать конфликт с помощью математических формул. 

\textbf{Опр:} Cтороны в "конфликте"  называются \textit{игроками}.

\textbf{Опр}: \textit{Множество игроков} обозначается как $I$ и каждый игрок принадлежит этому множеству:
$$i \in I \eqno (1)$$ 

\textbf{Опр}: \textit{Стратегия} в такой игре - правило(отображение), которое преписывает игроку для каждой ситуации в игре ход в это ситуации.

\textbf{Опр}: $S_i$ -\textit{Множество стратегий}, для каждого игрока $i$ своя стратегия: 
$$ \{S_i\}_{i \in I } \eqno (2)$$

\textbf{Опр}: \textit{Ситуация} - результат выбора игроками своих стратегий. 

\textbf{Опр}: Размер выигрыша определяется \textit{платежной функцией} - функция, оценивающая ту или иную ситуацию для отдельного игрока. Данная функция отображает ситуацию в число:
$$ i: \text{ } \{H_i\}_{i \in I } \eqno (3)$$ 
$$ H_{i} (s_1,s_2, ..., s_n) \in \mathbb{R} \eqno (4)$$
т.е каждый игрок оценивает ситуацию вещественным числом.

\textbf{Опр}: Множество игроков, множество стратегий и множество платежных функций называется \textit{игрой}:
$$ <i \in I, \{S_i\}_{i \in I },\{H_i\}_{i \in I } > \eqno (5)$$
 
\textbf{Пример:} на столе лежит $100$ камешков, играют два человека. Ход состоит в том, что каждый игрок забирает из кучки от $1$ до $5$ камешкев по своему усмотрению. Тот, кто взял последним, выиграл. Существуют ли выигрышные стратегии для игроков?

\textbf{Решение:}

Первый ход: берем 4 камня, а после дополняем количество камешкев до 6. Первый выигрывает.
$\blacksquare$

Для каждой из игр строится \textit{дерево игры}, состоящее из стратегий, где каждая \textit{ветвь} - отдельная игра, а \textit{узлы} данного вида - ситуации.

На основе дерева игры попытаемся создать \textit{матрицу данной игры} размером $m \times n$ Количество \textit{строк} в данной матрице - \textit{количество стратегий} первого игрока, количество \textit{столбцов} - количество стратегий второго игрока. 

Первый игрок выбирает какую-то строку этой матрицы, второй - какой-то столбец, другими словами первый игрок выбирает какую-то стратегию, а на пересечении столбцов и строк находится размер выигрыша первого игрока (при нулевом балансе у проигравшего получается $-1$, а у выигрывшего $1$).

Пусть $H_1$ - матрица выигрыша первого игрока, $H_2$ - матрица выигрыша второго игрока и сумма элементов на одинаковых позициях в этих таблицах равна нулю.

\textbf{Опр}: Если сумма платежных функций (матриц функций) равна нулю, то такая игра называется игрой с \textit{нулевой суммой}.
$$\sum_{i \in I}{H_{i} (s_1,s_2, ..., s_n) = 0} \eqno (6)$$ 

\textbf{Опр}: Если сумма равна какой-то константе, то такая игра называется игрой с \textit{постоянной суммой}.
$$\sum_{i \in I}{H_{i} (s_1,s_2, ..., s_n) = Const} \eqno (7)$$ 

\textbf{Опр}: \textit{антагонистическая игра} - игра двух игроков с нулевой суммой. В такой игре если выигрывает один, то обязательно проигрывает другой.

Так как сумма матриц равна нулю, то: 
$$H_1 = -H_2 \eqno (8)$$

следовательно, нам не нужно две матрицы и будем проводить рассуждение на основе матрицы выигрышей первого игрока.
\newpage
\subsection{Матричные игры}

Рассмотрим матрицу $ A_{m \times n}$ с элементами , являющимися вещественными числам, для антагонистической игры, в которую играют два игрока. Первый игрок выбирает номер строки, а второй игрок выбирает номер столбца. 

То, что находится на пересечении $(a_{ij})$ - размер выигрыша(проигрыша) игрока первого игрока, $(-a_{ij})$ - проигрыша(выигрыша) второго игрока.

Будем выписывать минимальные элементы по строке: 
$$ \underset{j}{\min} = \{a_{1,j_1}, ... , a_{m,j_m}\} \eqno (9)$$

\textbf{Опр}: Среди данных минимумов выберем $\max$ среди $\min$. Данная величина называется \textit{максимином}:
$$ \underset{i}{\max} (\text{ } \underset{j}{\min} \text{ } a_{ij}) = a_{i_{0},j_{0}} \eqno (10)$$

То есть в самой худшей ситуации, если он выберет эту строку, то это будет минимальным его выигрышем.

Допустим второй игрок выбирает первый столбец, тогда худшим вариантом для него будет максимум по строкам в каждом столбце:

$$ \underset{i}{\max} = \{a_{i_1,1}, ... , a_{i_n,n}\} \eqno (11)$$

\textbf{Опр}: Среди данных минимумов выберем $\min$ среди $\max$ (лучшее среди худшего). Данная величина называется \textit{минимаксом}:

$$ \underset{j}{\min} (\text{ }\underset{i}{\max} \text{ } a_{ij}) = a_{i_1,j_1} \eqno (12)$$

Получили гарантированный проигрыш второго игрока. 
Предположим, что эти элементы совпали, тогда такой элемент называется седловой точкой.

%\textbf{Опр}: седловой точкой называется точка, для которой $a_{i_0,j_0} = a_{i_1,j_1}$, являющаяся минимумом по одной оси, и точка максимума по другой.

\textbf{Опр}: Седловой точкой называется точка(элемент матрицы), которая является минимальным в своей строке и максимальной в своем столбце.

\textbf{Опр}: Устойчивая ситуация - ситуация, из которой невыгодно выходить любому игроку. Признак решения конфликта - наличие свойства устойчивости. 

Такое решение называется решением по \textit {Нэшу}.
\newpage

\textbf{Теорема 1}: (\textit{неравенство максимина и минимакса})

Дана матрица $A_{m \times n}$ и $a_{ij}$ - элементы матрицы. 

Рассмотрим максимин и минимакс: $a_{pq}$ и $a_{rs}$, такие, что:
$$ a_{pq} = \underset{i}{\max} \text{ } (\underset{j}{\min} \text{ } a_{ij}) \eqno (13)$$
$$ a_{rs} = \underset{j}{\min} \text{ } (\underset{i}{\max} \text{ } a_{ij}) \eqno (14)$$

Тогда $$a_{pq} \leq a_{rs} \eqno (15)$$

\textit{Доказательство}:

Рассмотрим матрицу и рассмотренные в ней элементы $a_{pq}$ и $ a_{rs}$.

$$\begin{pmatrix}
	
 \cdot  & \cdot &  \cdot  & \cdot  & \cdot \\ 
 \cdot &a_{pq} & \cdot  &a_{ps} & \cdot \\ 
 \cdot  & \cdot  & \cdot  & \cdot  & \cdot \\ 
 \cdot & \cdot & \cdot & a_{rs} &  \cdot \\ 
 \cdot & \cdot & \cdot & \cdot & \cdot 

\end{pmatrix}$$

Пусть рассматривается $a_{ps}$ - элемент матрицы $A_{m \times n}$.
$a_{ps} \leq a_{rs}$, так как $a_{rs}$ - максимум в столбце. 

С другой стороны $a_{pq}$ - минимум в строке, следовательно $a_{pq} \leq a_{ps}$. Тогда из двух неравенств получаем: $a_{pq} \leq a_{ps} \leq a_{rs}$. $\blacksquare$

Рассмотрим теперь теорему и необходимом и достаточном условии седловой точке в матрице.

\textbf{Теорема 2}: (\textit{необходимое и достаточное условие седловой точки})

Чтобы задача имела седловую точку необходимо и достаточно, чтобы $$a_{pq} = a_{rs} \eqno (16) $$

\textit{Доказательство}:

1. $\exists$  седловая чтока $\Rightarrow a_{pq} = a_{rs}$. Пусть $ a_{kl}$ - седловая точка.
$$\begin{pmatrix}
	
 \cdot  & \cdot &  \cdot &  \cdot  & \cdot \\ 
 \cdot &a_{pq} & \cdot  & \cdot  & \cdot \\ 
 \cdot  & \cdot  & a_{kl} &  a_{ks} &  \cdot \\ 
 \cdot & \cdot  &  \cdot & a_{rs} &  \cdot \\ 
 \cdot & \cdot  &  \cdot  &   \cdot    &  \cdot 

\end{pmatrix}$$
\newpage

Если бы мы писали строку максимумов, то в ней бы были точки $a_{kl},..., a_{rs}$, но в этой строке $a_{rs}$ является минимумом, следовательно: $$ a_{kl} \geq a_{rs} $$

Если бы мы писали столбец минимумов, то в ней бы были точки $a_{pq},..., a_{kl}$, но в этом столбце $a_{pq}$ является максимумом, следовательно: 
$$ a_{pq} \geq a_{kl}$$

Из двух неравенств получаем следующие соотношения:
$$ a_{pq} \geq a_{kl} \geq a_{rs} $$
$$ a_{pq} \geq a_{rs} $$

Но по формуле (15):
$$ a_{pq} \leq a_{rs} $$

Следовательно:
$$ a_{pq} = a_{rs} $$

Необходимость доказана

2. $a_{pq} = a_{rs}$  $\Rightarrow $ Нужно доказать, что $\exists$ - седловая точка 

Для доказательства обратного случая нужно построить каким-то образом седловую точку. Выберем точку $a_{ps}$, как показано ниже, и докажем, что данная точка является седловой. 
$$\begin{pmatrix}
	
  \cdot  &  \cdot &   \cdot &   \cdot  &  \cdot \\ 
  \cdot &a_{pq} &  \cdot  & a_{ps}  &  \cdot\\ 
  \cdot  &  \cdot  &  \cdot &   \cdot &   \cdot \\ 
  \cdot &  \cdot  &   \cdot & a_{rs} & \cdot \\ 
  \cdot &  \cdot  &   \cdot  &    \cdot    &   \cdot

\end{pmatrix}$$
$$ a_{pq} \leq a_{ps} \leq a_{rs}$$

Можно записать как равенство, так как по условию достаточности:
$$ a_{pq} = a_{ps} = a_{rs}$$

Этот элемент равен минимальному в строке и максимальному в столбце - определение минимакса и максимина $\underset{\text{def}}{\Rightarrow}$ седловая точка. $\blacksquare$

\textbf{Теорема 3}: (\textit{множество седловых точек})

Пусть $a_{kl}$ и $a_{uv}$ - седловые точки, тогда $a_{kv}$ и $a_{ul}$ - тоже седловые точки. 

\textit{Доказательство}:
$$\begin{pmatrix}
  \cdot  &  \cdot &   \cdot &   \cdot  &  \cdot \\ 
  \cdot &a_{kl} &  \cdot  & a_{kv}  &  \cdot \\ 
  \cdot  &  \cdot  &  \cdot &   \cdot &   \cdot \\ 
  \cdot & a_{ul}  &   \cdot & a_{uv} & \cdot \\ 
  \cdot &  \cdot  &   \cdot  &    \cdot    &   \cdot 
\end{pmatrix}$$
$$ a_{kl} \geq a_{ul} \geq a_{uv} \geq a_{kv} \geq a_{kl} $$

Так как концы равны, то можно заменить равенствами. 
$$ a_{kl} =a_{ul} = a_{uv} = a_{kv} = a_{kl} $$

Следовательно, $a_{ul}$ и $a_{kv}$ - максимальный в своем столбце и минимальный в своем столбце $\underset{\text{def}}{\Rightarrow}$ седловые точки. $\blacksquare$

\textbf{Замечание:} все седловые точки \textit{равны друг другу}.

\textbf{Замечание:} если элемент матрицы \textit{равен седловой точке}, то он \textit{не обязательно является седловой точкой}.

Рассмотрим пример:

$$
\begin{pmatrix}
1_{s} & 1_{s} \\
0 & 1 \\
\end{pmatrix}
$$

В данном примере в первой строке являются седловыми точками, но единица во второй строке не седловая точка, хоть и равна ей.

\section{17.02.2020}
Будем рассматривать переменные на прямоугольнике.

Пусть задана функция $f(x.y)$ и играется в антоганистическую игру двумя соперниками. 
Первый игрок выбирает значение $x$ на своем промежутке $x:a \leq x \leq b$, а второй игрок выбирает какое-то значение $y: c \leq y \leq d$. 

Требуется определить механизим седловой точки.\newpage
\textbf{Алгоритм нахождения седловой точки:}

\begin{enumerate}
  \item $ \underset{x}{\max} \text{ }\underset{y}{\max} \text{ }  f(x,y) $
  \begin{enumerate}
    \item $ \underset{y}{\min} \text{ } f(x,y) = g(x) $
    \item $ \underset{x}{\max} \text{ } g(x) = A $
  \end{enumerate}
  \item $ \underset{y}{\min} \text{ }\underset{x}{\max} \text{ }  f(x,y) $
  \begin{enumerate}
    \item $ \underset{x}{\max} \text{ } f(x,y) = h(y) $
    \item $ \underset{y}{\min} \text{ } h(y) = B $
  \end{enumerate}
  \item $A = B ?$

\end{enumerate}

Задача:
$$ f(x,y) = (x-y)^2 $$
$$ -1 \leq x \leq 1,-1 \leq y \leq 1 $$

\textbf{Решение:}
Найдем максимин:
$$ \underset{y}{\min} \text{ } f(x,y) = (f(x,y))_y'  = -2x + 2y = g(x)$$
$$ y = x $$
$$ A = \underset{x}{\max} \text{ } g(x) = 0 $$

Найдем минимакс:
$$ \underset{x}{\max} \text{ } f(x,y) = (f(x,y))_x'  = 2x -  2y = h(y) \Rightarrow x = y$$

При каждом y свое значение максимума:

Исследуем границы:
$$ \max (1-y)^2 = 4 \text{ ;  } \max (-1-y)^2 = 4 $$
$$h(y) = \left\{\begin{matrix}
(1-y)^2, -1 \leq y \leq 0\\ 
(1+y)^2, 0 \leq y \leq 1
\end{matrix}\right.$$
$$ B = \underset{y}{\min} \text{ } h(y) = 1 $$

Седловой точки нет, так как $A \neq B$

\section{02.03.2020}

Задача 2:
$$ f(x,y) = (x-y)^2 - 0.5y $$
$$ -1 \leq x \leq 1,-1 \leq y \leq 1 $$

\textbf{Решение:}
В общем случае, нет стационарной точки, значит будем искать на границах.

Найдем максимин:
$$ \underset{y}{\min} \text{ } f(x,y) = (f(x,y))_y'  = -2x + 2y -0.5 = g(x)$$
$$(f(x,y))_y'' = 2  -\text{точка минимума}$$ 
$$ y = x + 0.25$$

Из трех функций нужно было бы найти min среди трех выражений $f(x,1),f(x,-1),-0.0625x$
$$g(x) = \max{f(x,1),f(x,-1),f(x,x+0.25)}$$
$$ g(x) = -0.0625 - 0.5 x,\quad  x\to -1 $$
$$ A = \underset{x}{\max} \text{ } g(x) = 0.4375 $$

Найдем минимакс:
$$ \underset{x}{\max} \text{ } f(x,y) = (f(x,y))_x'  = 2x -  2y = h(y) \Rightarrow x = y$$

При каждом y свое значение максимума:

Исследуем границы:
$$ \max (1-y)^2 = 4 \text{ ;  } \max (-1-y)^2 = 4 $$
$$h(y) = \left\{\begin{matrix}
(1-y)^2, -1 \leq y \leq 0\\ 
(1+y)^2, 0 \leq y \leq 1
\end{matrix}\right.$$
$$ B = \underset{y}{\min} \text{ } h(y) = 1 $$

Седловой точки нет, так как $A \neq B$

Задача 3:
$$ f(x,y) = (x-y(1-y^2))^2 = x^2 - 2 x y + y^2 + 2 x y^3 - 2 y^4 + y^6$$
$$ -1 \leq x \leq 1,-1 \leq y \leq 1 $$

\textbf{Решение:}

Обозначим за $y(1-y^2) = u$, тогда получим функцию: $f(x,u) = (x-u)^2$

Для того, чтобы узнать границы, найдем минимум и максимум функции $u$:
$$ \min (y - y^3) =-\frac{2}{3\sqrt3}$$
$$ \max (y - y^3) = \frac{2}{3\sqrt3}$$

1. Найдем максимин:
$$ \underset{u}{\min} \text{ } f(x,u) = (f(x,u))_u'  = -2(x-u) - \text{точка минимума}$$
$$u=x$$
$$g(x) = \left \{
\begin{matrix}
(x+\frac{2}{3\sqrt3})^2 \quad -1 \leq x \leq -\frac{2}{3\sqrt3} \\
{}\\
  0 \qquad -\frac{2}{3\sqrt3} \leq x \leq \frac{2}{3\sqrt3} \\
{} \\
(x-\frac{2}{3\sqrt3})^2 \quad \frac{2}{3\sqrt3 } \leq x \leq 1
\end{matrix}\right.$$

$$ A = \underset{x}{\max} \text{ } g(x) = \left (1 - \frac{2}{3\sqrt {3}}\right)^2$$

2. Найдем минимакс:
$$ \underset{x}{\max} \text{ } f(x,u) = (f(x,u))_x'  = 2x -  2u = h(u) \Rightarrow x = u$$

Исследуем границы:
$$h(u) = \left\{
\begin{matrix}
(1+u)^2,\frac{2}{3\sqrt{3}}\geq u \geq 0 \\ 
(1-u)^2, -\frac{2}{3\sqrt{3}}\leq u \leq 0
\end{matrix}\right.$$

Рисуем график и находим минимум.
$$ B = \underset{u}{\min} \text{ } h(u) = 1 $$

Седловой точки нет, так как $A \neq B$
\end{document}
