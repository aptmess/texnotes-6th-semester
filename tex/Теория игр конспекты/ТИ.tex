\RequirePackage{ifluatex}
\let\ifluatex\relax

\documentclass[aps,%
12pt,%
final,%
oneside,
onecolumn,%
musixtex, %
superscriptaddress,%
centertags]{article} %% 
\topmargin=-40pt
\textheight=650pt
\usepackage[english,russian]{babel}
\usepackage[utf8]{inputenc}
%всякие настройки по желанию%
\usepackage[colorlinks=true,linkcolor=black,unicode=true]{hyperref}
\usepackage{euscript}
\usepackage{supertabular}
\usepackage[pdftex]{graphicx}
\usepackage{amsthm,amssymb, amsmath}
\usepackage{textcomp}
\usepackage[noend]{algorithmic}
\usepackage[ruled]{algorithm}
\usepackage{lipsum}
\usepackage{indentfirst}
\usepackage{babel}
\usepackage{pgfplots}
\pgfplotsset{compat=1.9}

\pgfplotsset{model/.style = {blue, samples = 100}}
\pgfplotsset{experiment/.style = {red}}

\selectlanguage{russian}

\setlength{\parindent}{2.4em}
\setlength{\parskip}{0.1em}
%\renewcommand{\baselinestretch}{2.0}

\usepackage{xcolor}
\usepackage{hyperref}
 
 % Цвета для гиперссылок
%\definecolor{linkcolor}{HTML}{799B03} % цвет ссылок
%\definecolor{urlcolor}{HTML}{799B03} % цвет гиперссылок
 
%\hypersetup{pdfstartview=FitH,  linkcolor=linkcolor,urlcolor=urlcolor, colorlinks=true}

\begin{document}

\begin{titlepage} 
\begin{center}
% Upper part of the page
%\textbf{\Large САНКТ-ПЕТЕРБУРГСКИЙ ГОСУДАРСТВЕННЫЙ ЭКОНОМИЧЕСКИЙ УНИВЕРСИТЕТ} \\[1.0cm]
%\textbf{\large Кафедра Прикладной Математики и Информатики}\\[3.5cm]
 
% Title
\textbf{}\\[10.0cm]
\textbf{\LARGE Теория игр}\\[0.5cm]
\textbf{\Large ПМ-1701} \\[0.1cm]

%supervisor
\begin{center} \large
{Преподаватель:} \\[0.5cm]
\textsc {Чернов Виктор Петрович}\\
{viktor\_chernov@mail.ru}\\
\end{center}
% \begin{flushright} \large
%\emph{Рецензент:} \\
%д.ф. - м.н., профессор \textsc{Надеемся Нам Помогут}
%\end{flushright}
%\begin{flushright} \large
%\emph{Заведующий кафедрой:} \\
%д.ф. - м.н., профессор \textsc{Не Обмани Себя}
%\end{flushright}
\vfill 

% Bottom of the page
{\large {Санкт-Петербург}} \par
{\large {2020 г., 6 семестр}}
\end{center} 
\end{titlepage}

% Table of contents
\begin{thebibliography}{3}
\bibitem{eliseeva}
Теория игр
\end{thebibliography}
\tableofcontents
\newpage

\section{10.02.2020}
\subsection{Введение}

Предметом теории игр является моделированием конфликтных ситуаций. Зачинателем "Теории игр" является Джон фон Нейман, а последователем является Джон Нэш.

Зададимся вопросом, а как описать конфликт с помощью математических формул. 

Опр: Стороны в "конфликте"  называются \textit{игроками}.

Опр: Множество игроков обозначается как $I$ и каждый игрок принадлежит этому множеству:
$$i \in I$$ 

Опр: $S_i$ - множество стратегий, для каждого игрока $i$ своя стратегия: 
$$ \{S_i\}_{i \in I }$$

Опр: \textit{Ситуация} - результат выбора игроками своих стратегий. 

Опр: Размер выигрыша определяется \textit{платежной функцией} - функция, оценивающая ту или иную ситуацию для отдельного игрока. Данная функция отображает ситуацию в число:
$$ i: \text{ } \{H_i\}_{i \in I }$$ 
$$ H_{i} (s_1,s_2, ..., s_n) \in \mathbb{R}$$
т.е каждый игрок оценивает ситуацию вещественным числом.

Опр: Множество игроков, множество стратегий и множество платежных функций называется \textit{игрой}:
$$ <i \in I, \{S_i\}_{i \in I },\{H_i\}_{i \in I } > $$
 
\textbf{Пример:} на столе лежит $100$ камешков, играют два человека. Ход состоит в том, что каждый игрок забирает из кучки от $1$ до $5$ камешкев по своему усмотрению. Тот, кто взял последним, выиграл. Существуют ли стратегии?

\textbf{Решение:}

Первый ход: берем 4 камня, а после дополняем количество камешкев до 6. Первый выигрывает.
$\blacksquare$

Опр: \textit{стратегия} в такой игре - правило(отображение), которое преписывает игроку для каждой ситуации в игре ход в это ситуации.
\newpage

Для каждой из игр строится дерево игры, состоящее из стратегий, где каждая ветвь - отдельная игра, а узлы данного вида - ситуации.

На основе дерева игры попытаемся создать \textit{матрицу данной игры} размером $m \times n$ Количество \textit{строк} в данной матрице - \textit{количество стратегий} первого игрока, количество столбцов - количество стратегий второго игрока. 

Первый игрок выбирает какую-то строку этой матрицы, второй - какой-то столбец, другими словами первый игрок выбирает какую-то стратегию, а на пересечении столбцов и строк находится размер выигрыша первого игрока (при нулевом балансе у проигравшего получается $-1$ рубль, а у выигрывшего - $+1$).

$H_1$ - матрица выигрыша первого игрока, $H_2$ - матрица выигрыша второго игрока и сумма элементов на одинаковых позициях в этих таблицах равна нулю.
Если сумма платежных функций (матриц функций) равна нулю, то такая игра называется игрой с \textit{нулевой суммой}.
$$\sum_{i \in I}{H_{i} (s_1,s_2, ..., s_n) = 0}$$ 

Если сумма равна какой-то константе, то такая игра называется игрой с \textit{постоянной суммой}.
$$\sum_{i \in I}{H_{i} (s_1,s_2, ..., s_n) = Const}$$ 

Опр: \textit{антагонистическая игра} - игра двух игроков с нулевой суммой. В такой игре если выигрывает один, то обязательно проигрывает другой.

Так как сумма матриц равна нулю, то: 
$$H_1 = -H_2 $$
следовательно, нам не нужно две матрицы и будем проводить рассуждение на основе матрицы выигрышей первого игрока.

\subsection{Матричные игры}

Рассмотрим матрицу $ A_{m \times n}$ с элементами , являющимися вещественными числам, для антагонистической игры, в которую играют два игрока. Первый игрок выбирает номер строки, а второй игрок выбирает номер столбца. 

То, что находится на пересечении $a_{ij}$ - размер выигрыша(проигрыша) игрока первого игрока, $-a_{ij}$ - проигрыша(выигрыша) второго игрока.

Будем выписывать минимальные элементы по строке: 
$$ \min = \{a_{1,j_1}, ... , a_{m,j_m}\}$$

Среди данных минимумов выберем $max$ среди $min$. Данная величина называется \textit{максимином}:
$$ \underset{i}{\max} \text{ } \underset{j}{\min} \text{ } a_{ij} = a_{i_0,j_0}$$

То есть в самой худшей ситуации, если он выберет эту строку, то это будет минимальным его выигрышем.

Допустим второй игрок выбирает первый столбец, тогда худшим вариантом для него будет максимум по строкам в каждом столбце:

$$ \max = \{a_{i_1,1}, ... , a_{i_n,n}\}$$

Среди данных минимумов выберем $min$ среди $max$(лучшее среди худшего). Данная величина называется \textit{минимаксом}:

$$ \underset{j}{\min} \text{ }\underset{i}{\max} \text{ } a_{ij} = a_{i_1,j_1} $$

Получили гарантированный проигрыш второго игрока. 
Предположим, что эти элементы совпали, то такой элемент называется седловой точкой.

Опр: седловой точкой называется точка, для которой $a_{i_0,j_0} = a_{i_1,j_1}$, являющаяся минимумом по одной оси, и точка максимума по другой.

Опр: седловой точкой называется точка(элемент матрицы), которая является минимальным в своей строке и максимальной в своем столбце.

Мы получили ситуацию, в которой ни одному из игроков не выгодно из ситуаиции выходить.

Если мы нашли устойчивую ситуацию (ситуацию, из которой невыгодно выходить любому игроку), то мы решили игру. Признак решения конфликта - наличия свойства устойчивости.

Такое решение называется решением по \textit {Нэшу}.

\textbf{Теорема 1}: (неравенство максимина и минимакса)

Дана матрица $A_{m \times n}$ и $a_{ij}$ - элементы матрицы. Рассмотрим максимин и минимакс: $a_{pq}$ и $a_{rs}$. Тогда $a_{pq} \leq a_{rs}$

\textbf{Док-во}:

Рассмотрим матрицу и рассмотренные в ней элементы $a_{pq}$ и $ a_{rs}$.

$$\begin{pmatrix}
	
 *  & * &  *  & *  & * \\ 
 * &a_{pq} &*  &a_{ps} &* \\ 
 *  & *  & *  & *  & * \\ 
 * & * & * & a_{rs} &  * \\ 
 * & * & * & * & * 

\end{pmatrix}$$

Рассмотрим элемент $a_{ps}$. $a_{ps} \leq a_{rs}$. С другой стороны $a_{pq}$ $\Rightarrow$ это минимум в строке, следовательно, он меньше либо равен $a_{ps}$. Теорема доказана. $\blacksquare$

\textbf{Теорема 2}: (необходимое и достаточное условие седловой точки)

Чтобы задача имела седловую точку необходимо и достаточно, чтобы $a_{pq} = a_{rs} $

\textbf{Док-во}:

1. $\exists$  седловая чтока $\Rightarrow a_{pq} = a_{rs}$. Пусть $ a_{kl}$ - седловая точка.


$$\begin{vmatrix}
	
 *  & * &  * &  *  & * \\ 
 * &a_{pq} & *  & *  & * \\ 
 *  & *  & a_{kl} &  a_{ks} &  * \\ 
 * & *  &  * & a_{rs} &* \\ 
 * & *  &  *  &   *    &  * 

\end{vmatrix}$$

Если бы мы писали строку максимумов, то в ней бы были точки $a_{kl} и a_{rs}$, но в этой строке $a_{rs}$ является минимумом, следовательно: $$ a_{kl} \geq a_{rs} $$

Если бы мы писали столбец минимумов, то в ней бы были точки $a_{pq} и a_{kl}$, но в этом столбце $a_{pq}$ является максимумом, следовательно: 
$$ a_{pq} \geq a_{kl}$$

Следовательно:
$$ a_{pq} \geq a_{kl} \geq a_{rs} $$
$$ a_{pq} \geq a_{rs} $$

Но:
$$ a_{pq} \leq a_{rs} $$

Следовательно:
$$ a_{pq} = a_{rs} $$

Доказано, что равенство выполняется. Необходимость доказана. Докажем достаточность.

2. $a_{pq} = a_{rs}$  $\Rightarrow $ Нужно доказать, что $\exists$ - седловая точка 

Рассмотрим $a_{ps}$

Попытаемся построить данную точку. Хочу доказать, что 

$$\begin{vmatrix}
	
 *  & * &  * &  *  & * \\ 
 * &a_{pq} & *  & a_{ps}  & * \\ 
 *  & *  & * &  * &  * \\ 
 * & *  &  * & a_{rs} &* \\ 
 * & *  &  *  &   *    &  * 

\end{vmatrix}$$

$$ a_{pq} \leq a_{ps} \leq a_{rs}$$

Можно записать как равенство, так как по условию достаточности:
$$ a_{pq} = a_{ps} = a_{rs}$$

Этот элемент равный минимальному в строке и максимальному в столбце - определение минимакса и максимина. Следовательно, по определению, это седловая точка.

Теорема доказана. $\blacksquare$


\textbf{Теорема 3}: (неравенство максимина и минимакса)

$a_{kl}$ и $a_{uv}$ - седловые точки. 
Тогда: $a_{kv}$ и $a_{ul}$ - тоже седловые точки. 

\textbf{Док-во}:
$$\begin{vmatrix}
	
 *  & * &  * &  *  & * \\ 
 * &a_{kl} & *  & a_{kv}  & * \\ 
 *  & *  & * &  * &  * \\ 
 * & a_{ul}  &  * & a_{uv} &* \\ 
 * & *  &  *  &   *    &  * 

\end{vmatrix}$$
$$ a_{kl} \geq a_{ul} \geq a_{uv} \geq a_{kv} \geq a_{kl} $$

Так как концы равны, то можно заменить равенствами. 
$$ a_{kl} =a_{ul} = a_{uv} = a_{kv} = a_{kl} $$

Следовательно, $a_{ul}$ $a_{kv}$ и - максимальный в своем столбце и минимальный в своем столбце, следовательно это седловые точки. $\blacksquare$

\textbf{Замечание:} все седловые точки равны друг другу.

\textbf{Замечание:} Если элемент матрицы равен седловой точке, то он не является седловой точкой.

Рассмотрим пример:

$$
\begin{vmatrix}
1_{s} & 1_{s} \\
0 & 1 \\
\end{vmatrix}
$$

В данном примере в первой строке являются седловыми точками, но единица во второй строке не седловая точка, хоть и равна ей.
\end{document}
