\RequirePackage{ifluatex}
\let\ifluatex\relax

\documentclass[aps,%
12pt,%
final,%
oneside,
onecolumn,%
musixtex, %
superscriptaddress,%
centertags]{article} %% 
\topmargin=-40pt
\textheight=650pt
\usepackage[english,russian]{babel}
\usepackage[utf8]{inputenc}
%всякие настройки по желанию%
\usepackage[colorlinks=true,linkcolor=black,unicode=true]{hyperref}
\usepackage{euscript}
\usepackage{supertabular}
\usepackage[pdftex]{graphicx}
\usepackage{amsthm,amssymb, amsmath}
\usepackage{textcomp}
\usepackage[noend]{algorithmic}
\usepackage[ruled]{algorithm}
\usepackage{lipsum}
\usepackage{indentfirst}
\usepackage{babel}
\usepackage{pgfplots}
\usepackage{setspace}
\linespread{1.15}
\pgfplotsset{compat=1.9}

\pgfplotsset{model/.style = {blue, samples = 100}}
\pgfplotsset{experiment/.style = {red}}

\selectlanguage{russian}

\setlength{\parindent}{2.4em}
\setlength{\parskip}{0.1em}
%\renewcommand{\baselinestretch}{2.0}

\usepackage{xcolor}
\usepackage{hyperref}
 
 % Цвета для гиперссылок
%\definecolor{linkcolor}{HTML}{799B03} % цвет ссылок
%\definecolor{urlcolor}{HTML}{799B03} % цвет гиперссылок
 
%\hypersetup{pdfstartview=FitH,  linkcolor=linkcolor,urlcolor=urlcolor, colorlinks=true}

\begin{document}

\begin{titlepage} 
\begin{center}
% Upper part of the page
%\textbf{\Large САНКТ-ПЕТЕРБУРГСКИЙ ГОСУДАРСТВЕННЫЙ ЭКОНОМИЧЕСКИЙ УНИВЕРСИТЕТ} \\[1.0cm]
%\textbf{\large Кафедра Прикладной Математики и Информатики}\\[3.5cm]
 
% Title
\textbf{}\\[10.0cm]
\textbf{\LARGE Теория игр}\\[0.5cm]
\textbf{\Large ПМ-1701} \\[0.1cm]

%supervisor
\begin{center} \large
{Преподаватель:} \\[0.5cm]
\textsc {Чернов Виктор Петрович}\\
{viktor\_chernov@mail.ru}\\
\end{center}
% \begin{flushright} \large
%\emph{Рецензент:} \\
%д.ф. - м.н., профессор \textsc{Надеемся Нам Помогут}
%\end{flushright}
%\begin{flushright} \large
%\emph{Заведующий кафедрой:} \\
%д.ф. - м.н., профессор \textsc{Не Обмани Себя}
%\end{flushright}
\vfill 

% Bottom of the page
{\large {Санкт-Петербург}} \par
{\large {2020 г., 6 семестр}}
\end{center} 
\end{titlepage}

% Table of contents
\begin{thebibliography}{3}
\bibitem{eliseeva}
Теория игр
\end{thebibliography}
\tableofcontents
\newpage

\section{10.02.2020}
\subsection{Введение}

Предметом теории игр является моделированием конфликтных ситуаций. Зачинателем "Теории игр" является Джон фон Нейман, а последователем является Джон Нэш. Зададимся вопросом, а как описать конфликт с помощью математических формул. 

\textbf{Опр:} Cтороны в "конфликте"  называются \textit{игроками}.

\textbf{Опр}: \textit{Множество игроков} обозначается как $I$ и каждый игрок принадлежит этому множеству:
$$i \in I \eqno (1)$$ 

\textbf{Опр}: \textit{Стратегия} в такой игре - правило(отображение), которое преписывает игроку для каждой ситуации в игре ход в это ситуации.

\textbf{Опр}: $S_i$ -\textit{Множество стратегий}, для каждого игрока $i$ своя стратегия: 
$$ \{S_i\}_{i \in I } \eqno (2)$$

\textbf{Опр}: \textit{Ситуация} - результат выбора игроками своих стратегий. 

\textbf{Опр}: Размер выигрыша определяется \textit{платежной функцией} - функция, оценивающая ту или иную ситуацию для отдельного игрока. Данная функция отображает ситуацию в число:
$$ i: \text{ } \{H_i\}_{i \in I } \eqno (3)$$ 
$$ H_{i} (s_1,s_2, ..., s_n) \in \mathbb{R} \eqno (4)$$
т.е каждый игрок оценивает ситуацию вещественным числом.

\textbf{Опр}: Множество игроков, множество стратегий и множество платежных функций называется \textit{игрой}:
$$ <i \in I, \{S_i\}_{i \in I },\{H_i\}_{i \in I } > \eqno (5)$$
 
\textbf{Пример:} на столе лежит $100$ камешков, играют два человека. Ход состоит в том, что каждый игрок забирает из кучки от $1$ до $5$ камешкев по своему усмотрению. Тот, кто взял последним, выиграл. Существуют ли выигрышные стратегии для игроков?

\textbf{Решение:}

Первый ход: берем 4 камня, а после дополняем количество камешкев до 6. Первый выигрывает.
$\blacksquare$

Для каждой из игр строится \textit{дерево игры}, состоящее из стратегий, где каждая \textit{ветвь} - отдельная игра, а \textit{узлы} данного вида - ситуации.

На основе дерева игры попытаемся создать \textit{матрицу данной игры} размером $m \times n$ Количество \textit{строк} в данной матрице - \textit{количество стратегий} первого игрока, количество \textit{столбцов} - количество стратегий второго игрока. 

Первый игрок выбирает какую-то строку этой матрицы, второй - какой-то столбец, другими словами первый игрок выбирает какую-то стратегию, а на пересечении столбцов и строк находится размер выигрыша первого игрока (при нулевом балансе у проигравшего получается $-1$, а у выигрывшего $1$).

Пусть $H_1$ - матрица выигрыша первого игрока, $H_2$ - матрица выигрыша второго игрока и сумма элементов на одинаковых позициях в этих таблицах равна нулю.

\textbf{Опр}: Если сумма платежных функций (матриц функций) равна нулю, то такая игра называется игрой с \textit{нулевой суммой}.
$$\sum_{i \in I}{H_{i} (s_1,s_2, ..., s_n) = 0} \eqno (6)$$ 

\textbf{Опр}: Если сумма равна какой-то константе, то такая игра называется игрой с \textit{постоянной суммой}.
$$\sum_{i \in I}{H_{i} (s_1,s_2, ..., s_n) = Const} \eqno (7)$$ 

\textbf{Опр}: \textit{антагонистическая игра} - игра двух игроков с нулевой суммой. В такой игре если выигрывает один, то обязательно проигрывает другой.

Так как сумма матриц равна нулю, то: 
$$H_1 = -H_2 \eqno (8)$$

следовательно, нам не нужно две матрицы и будем проводить рассуждение на основе матрицы выигрышей первого игрока.
\newpage
\subsection{Матричные игры}

Рассмотрим матрицу $ A_{m \times n}$ с элементами , являющимися вещественными числам, для антагонистической игры, в которую играют два игрока. Первый игрок выбирает номер строки, а второй игрок выбирает номер столбца. 

То, что находится на пересечении $(a_{ij})$ - размер выигрыша(проигрыша) игрока первого игрока, $(-a_{ij})$ - проигрыша(выигрыша) второго игрока.

Будем выписывать минимальные элементы по строке: 
$$ \underset{j}{\min} = \{a_{1,j_1}, ... , a_{m,j_m}\} \eqno (9)$$

\textbf{Опр}: Среди данных минимумов выберем $\max$ среди $\min$. Данная величина называется \textit{максимином}:
$$ \underset{i}{\max} (\text{ } \underset{j}{\min} \text{ } a_{ij}) = a_{i_{0},j_{0}} \eqno (10)$$

То есть в самой худшей ситуации, если он выберет эту строку, то это будет минимальным его выигрышем.

Допустим второй игрок выбирает первый столбец, тогда худшим вариантом для него будет максимум по строкам в каждом столбце:

$$ \underset{i}{\max} = \{a_{i_1,1}, ... , a_{i_n,n}\} \eqno (11)$$

\textbf{Опр}: Среди данных минимумов выберем $\min$ среди $\max$ (лучшее среди худшего). Данная величина называется \textit{минимаксом}:

$$ \underset{j}{\min} (\text{ }\underset{i}{\max} \text{ } a_{ij}) = a_{i_1,j_1} \eqno (12)$$

Получили гарантированный проигрыш второго игрока. 
Предположим, что эти элементы совпали, тогда такой элемент называется седловой точкой.

%\textbf{Опр}: седловой точкой называется точка, для которой $a_{i_0,j_0} = a_{i_1,j_1}$, являющаяся минимумом по одной оси, и точка максимума по другой.

\textbf{Опр}: Седловой точкой называется точка(элемент матрицы), которая является минимальным в своей строке и максимальной в своем столбце.

\textbf{Опр}: Устойчивая ситуация - ситуация, из которой невыгодно выходить любому игроку. Признак решения конфликта - наличие свойства устойчивости. 

Такое решение называется решением по \textit {Нэшу}.
\newpage

\textbf{Теорема 1}: (\textit{неравенство максимина и минимакса})

Дана матрица $A_{m \times n}$ и $a_{ij}$ - элементы матрицы. 

Рассмотрим максимин и минимакс: $a_{pq}$ и $a_{rs}$, такие, что:
$$ a_{pq} = \underset{i}{\max} \text{ } (\underset{j}{\min} \text{ } a_{ij}) \eqno (13)$$
$$ a_{rs} = \underset{j}{\min} \text{ } (\underset{i}{\max} \text{ } a_{ij}) \eqno (14)$$

Тогда $$a_{pq} \leq a_{rs} \eqno (15)$$

\textit{Доказательство}:

Рассмотрим матрицу и рассмотренные в ней элементы $a_{pq}$ и $ a_{rs}$.

$$\begin{pmatrix}
	
 \cdot  & \cdot &  \cdot  & \cdot  & \cdot \\ 
 \cdot &a_{pq} & \cdot  &a_{ps} & \cdot \\ 
 \cdot  & \cdot  & \cdot  & \cdot  & \cdot \\ 
 \cdot & \cdot & \cdot & a_{rs} &  \cdot \\ 
 \cdot & \cdot & \cdot & \cdot & \cdot 

\end{pmatrix}$$

Пусть рассматривается $a_{ps}$ - элемент матрицы $A_{m \times n}$.
$a_{ps} \leq a_{rs}$, так как $a_{rs}$ - максимум в столбце. 

С другой стороны $a_{pq}$ - минимум в строке, следовательно $a_{pq} \leq a_{ps}$. Тогда из двух неравенств получаем: $a_{pq} \leq a_{ps} \leq a_{rs}$. $\blacksquare$

Рассмотрим теперь теорему и необходимом и достаточном условии седловой точке в матрице.

\textbf{Теорема 2}: (\textit{необходимое и достаточное условие седловой точки})

Чтобы задача имела седловую точку необходимо и достаточно, чтобы $$a_{pq} = a_{rs} \eqno (16) $$

\textit{Доказательство}:

1. $\exists$  седловая чтока $\Rightarrow a_{pq} = a_{rs}$. Пусть $ a_{kl}$ - седловая точка.
$$\begin{pmatrix}
	
 \cdot  & \cdot &  \cdot &  \cdot  & \cdot \\ 
 \cdot &a_{pq} & \cdot  & \cdot  & \cdot \\ 
 \cdot  & \cdot  & a_{kl} &  a_{ks} &  \cdot \\ 
 \cdot & \cdot  &  \cdot & a_{rs} &  \cdot \\ 
 \cdot & \cdot  &  \cdot  &   \cdot    &  \cdot 

\end{pmatrix}$$
\newpage

Если бы мы писали строку максимумов, то в ней бы были точки $a_{kl},..., a_{rs}$, но в этой строке $a_{rs}$ является минимумом, следовательно: $$ a_{kl} \geq a_{rs} $$

Если бы мы писали столбец минимумов, то в ней бы были точки $a_{pq},..., a_{kl}$, но в этом столбце $a_{pq}$ является максимумом, следовательно: 
$$ a_{pq} \geq a_{kl}$$

Из двух неравенств получаем следующие соотношения:
$$ a_{pq} \geq a_{kl} \geq a_{rs} $$
$$ a_{pq} \geq a_{rs} $$

Но по формуле (15):
$$ a_{pq} \leq a_{rs} $$

Следовательно:
$$ a_{pq} = a_{rs} $$

Необходимость доказана

2. $a_{pq} = a_{rs}$  $\Rightarrow $ Нужно доказать, что $\exists$ - седловая точка 

Для доказательства обратного случая нужно построить каким-то образом седловую точку. Выберем точку $a_{ps}$, как показано ниже, и докажем, что данная точка является седловой. 
$$\begin{pmatrix}
	
  \cdot  &  \cdot &   \cdot &   \cdot  &  \cdot \\ 
  \cdot &a_{pq} &  \cdot  & a_{ps}  &  \cdot\\ 
  \cdot  &  \cdot  &  \cdot &   \cdot &   \cdot \\ 
  \cdot &  \cdot  &   \cdot & a_{rs} & \cdot \\ 
  \cdot &  \cdot  &   \cdot  &    \cdot    &   \cdot

\end{pmatrix}$$
$$ a_{pq} \leq a_{ps} \leq a_{rs}$$

Можно записать как равенство, так как по условию достаточности:
$$ a_{pq} = a_{ps} = a_{rs}$$

Этот элемент равен минимальному в строке и максимальному в столбце - определение минимакса и максимина $\underset{\text{def}}{\Rightarrow}$ седловая точка. $\blacksquare$

\textbf{Теорема 3}: (\textit{множество седловых точек})

Пусть $a_{kl}$ и $a_{uv}$ - седловые точки, тогда $a_{kv}$ и $a_{ul}$ - тоже седловые точки. 

\textit{Доказательство}:
$$\begin{pmatrix}
  \cdot  &  \cdot &   \cdot &   \cdot  &  \cdot \\ 
  \cdot &a_{kl} &  \cdot  & a_{kv}  &  \cdot \\ 
  \cdot  &  \cdot  &  \cdot &   \cdot &   \cdot \\ 
  \cdot & a_{ul}  &   \cdot & a_{uv} & \cdot \\ 
  \cdot &  \cdot  &   \cdot  &    \cdot    &   \cdot 
\end{pmatrix}$$
$$ a_{kl} \geq a_{ul} \geq a_{uv} \geq a_{kv} \geq a_{kl} $$

Так как концы равны, то можно заменить равенствами. 
$$ a_{kl} =a_{ul} = a_{uv} = a_{kv} = a_{kl} $$

Следовательно, $a_{ul}$ и $a_{kv}$ - максимальный в своем столбце и минимальный в своем столбце $\underset{\text{def}}{\Rightarrow}$ седловые точки. $\blacksquare$

\textbf{Замечание:} все седловые точки \textit{равны друг другу}.

\textbf{Замечание:} если элемент матрицы \textit{равен седловой точке}, то он \textit{не обязательно является седловой точкой}.

Рассмотрим пример:

$$
\begin{pmatrix}
1_{s} & 1_{s} \\
0 & 1 \\
\end{pmatrix}
$$

В данном примере в первой строке являются седловыми точками, но единица во второй строке не седловая точка, хоть и равна ей.

\section{17.02.2020}
Будем рассматривать переменные на прямоугольнике.

Пусть задана функция $f(x.y)$ и играется в антоганистическую игру двумя соперниками. 
Первый игрок выбирает значение $x$ на своем промежутке $x:a \leq x \leq b$, а второй игрок выбирает какое-то значение $y: c \leq y \leq d$. 

Требуется определить механизим седловой точки.\newpage
\textbf{Алгоритм нахождения седловой точки:}

\begin{enumerate}
  \item $ \underset{x}{\max} \text{ }\underset{y}{\max} \text{ }  f(x,y) $
  \begin{enumerate}
    \item $ \underset{y}{\min} \text{ } f(x,y) = g(x) $
    \item $ \underset{x}{\max} \text{ } g(x) = A $
  \end{enumerate}
  \item $ \underset{y}{\min} \text{ }\underset{x}{\max} \text{ }  f(x,y) $
  \begin{enumerate}
    \item $ \underset{x}{\max} \text{ } f(x,y) = h(y) $
    \item $ \underset{y}{\min} \text{ } h(y) = B $
  \end{enumerate}
  \item $A = B ?$

\end{enumerate}

Задача:
$$ f(x,y) = (x-y)^2 $$
$$ -1 \leq x \leq 1,-1 \leq y \leq 1 $$

\textbf{Решение:}
Найдем максимин:
$$ \underset{y}{\min} \text{ } f(x,y) = (f(x,y))_y'  = -2x + 2y = g(x)$$
$$ y = x $$
$$ A = \underset{x}{\max} \text{ } g(x) = 0 $$

Найдем минимакс:
$$ \underset{x}{\max} \text{ } f(x,y) = (f(x,y))_x'  = 2x -  2y = h(y) \Rightarrow x = y$$

При каждом y свое значение максимума:

Исследуем границы:
$$ \max (1-y)^2 = 4 \text{ ;  } \max (-1-y)^2 = 4 $$
$$h(y) = \left\{\begin{matrix}
(1-y)^2, -1 \leq y \leq 0\\ 
(1+y)^2, 0 \leq y \leq 1
\end{matrix}\right.$$
$$ B = \underset{y}{\min} \text{ } h(y) = 1 $$

Седловой точки нет, так как $A \neq B$

\section{02.03.2020}

Задача 2:
$$ f(x,y) = (x-y)^2 - 0.5y $$
$$ -1 \leq x \leq 1,-1 \leq y \leq 1 $$

\textbf{Решение:}
В общем случае, нет стационарной точки, значит будем искать на границах.

Найдем максимин:
$$ \underset{y}{\min} \text{ } f(x,y) = (f(x,y))_y'  = -2x + 2y -0.5 = g(x)$$
$$(f(x,y))_y'' = 2  -\text{точка минимума}$$ 
$$ y = x + 0.25$$

Из трех функций нужно было бы найти min среди трех выражений $f(x,1),f(x,-1),-0.0625x$
$$g(x) = \max{f(x,1),f(x,-1),f(x,x+0.25)}$$
$$ g(x) = -0.0625 - 0.5 x,\quad  x\to -1 $$
$$ A = \underset{x}{\max} \text{ } g(x) = 0.4375 $$

Найдем минимакс:
$$ \underset{x}{\max} \text{ } f(x,y) = (f(x,y))_x'  = 2x -  2y = h(y) \Rightarrow x = y$$

При каждом y свое значение максимума:

Исследуем границы:
$$ \max (1-y)^2 = 4 \text{ ;  } \max (-1-y)^2 = 4 $$
$$h(y) = \left\{\begin{matrix}
(1-y)^2, -1 \leq y \leq 0\\ 
(1+y)^2, 0 \leq y \leq 1
\end{matrix}\right.$$
$$ B = \underset{y}{\min} \text{ } h(y) = 1 $$

Седловой точки нет, так как $A \neq B$

Задача 3:
$$ f(x,y) = (x-y(1-y^2))^2 = x^2 - 2 x y + y^2 + 2 x y^3 - 2 y^4 + y^6$$
$$ -1 \leq x \leq 1,-1 \leq y \leq 1 $$

\textbf{Решение:}

Обозначим за $y(1-y^2) = u$, тогда получим функцию: $f(x,u) = (x-u)^2$

Для того, чтобы узнать границы, найдем минимум и максимум функции $u$:
$$ \min (y - y^3) =-\frac{2}{3\sqrt3}$$
$$ \max (y - y^3) = \frac{2}{3\sqrt3}$$

1. Найдем максимин:
$$ \underset{u}{\min} \text{ } f(x,u) = (f(x,u))_u'  = -2(x-u) - \text{точка минимума}$$
$$u=x$$
$$g(x) = \left \{
\begin{matrix}
(x+\frac{2}{3\sqrt3})^2 \quad -1 \leq x \leq -\frac{2}{3\sqrt3} \\
{}\\
  0 \qquad -\frac{2}{3\sqrt3} \leq x \leq \frac{2}{3\sqrt3} \\
{} \\
(x-\frac{2}{3\sqrt3})^2 \quad \frac{2}{3\sqrt3 } \leq x \leq 1
\end{matrix}\right.$$

$$ A = \underset{x}{\max} \text{ } g(x) = \left (1 - \frac{2}{3\sqrt {3}}\right)^2$$

2. Найдем минимакс:
$$ \underset{x}{\max} \text{ } f(x,u) = (f(x,u))_x'  = 2x -  2u = h(u) \Rightarrow x = u$$

Исследуем границы:
$$h(u) = \left\{
\begin{matrix}
(1+u)^2,\frac{2}{3\sqrt{3}}\geq u \geq 0 \\ 
(1-u)^2, -\frac{2}{3\sqrt{3}}\leq u \leq 0
\end{matrix}\right.$$

Рисуем график и находим минимум.
$$ B = \underset{u}{\min} \text{ } h(u) = 1 $$

Седловой точки нет, так как $A \neq B$.


\subsubsection{23.03.2020 Смешанное расширение матричной игры}

Мы видели на разных примерах, что матрица может иметь седловую точку (седловой элемент), а может не иметь ее. Соответственно, решение игры, определяющее равновесие по Нзшу, может существовать, а может не существовать. 

Оказывается, что в некотором расширенном смысле решение игры существует всегда.

Рассмотрим матрицу $A$ размерности $m \times n$. Ее элементы обозначим как $a_ij$.

\textit{Смешанной стратегией} первого игрока назовем распределение вероятностей выбора той или иной строки. Таким образом, смешанная стратегия – это вектор $Р$ вида:
$$ P = (p_1,p_2,\ldots,p_m), (1 \times m)$$
$$ \sum_{i=1}^n p_i = 1$$

$p_i$ - вероятность выбора игроком его $i$-ой стратегии.

все компоненты которого неотрицательны и сумма компонент которого равна 1. 

Аналогично определяется понятие смешанной стратегии второго игрока. Она представляет собой вектор Q вида:
$$Q = (q_1,q_2,\ldots, q_n), (1 \times n)$$

все компоненты которого неотрицательны и сумма компонент которого равна 1.

Множество смешанных стратегий бесконечно. Частный случай смешанной стратегии, когда одна компонента равна 1, а остальные равны 0, соответствует выбору конкретной строки (или столбца), то есть соответствует стратегии в прежнем смысле.  

Теперь такие стратегии мы будем называть чистыми стратегиями. Таким образом, чистые стратегии – частный случай смешанных. Множество чистых стратегий конечно.

Предположим, игроки выбрали какие-то свои смешанные стратегии Р и Q. Представим себе, что игра с матрицей А разыгрывается многократно. В каждом конкретном розыгрыше игроки выбирают строку и столбец случайным образом, но каждый игрок делает такой выбор в соответствии с распределением вероятностей Р (для первого игрока) и Q (для второго). 

Результатом такой игры считается математическое ожидание величины выигрыша. Это математическое ожидание можно задать в виде произведения
$$ \mathbb{E} = PAQ^T \eqno (18)$$

Такая игра в новом смысле, подразумевающая возможность использования смешанных стратегий, называется \textit{смешанным расширением первоначальной игры.}

Пара смешанных стратегий $P^*,Q^*$ является \textit{седловой точкой смешанного расширения}, если для любых смешанных стратегий Р и Q выполнены неравенства:

$$PAQ^{*T} \leq P^*AQ^{*T} \leq P^*AQ^T \eqno (19)$$

Эта пара неравенств показывает, что первому игроку невыгодно отклоняться от своей стратегии $Р^*$ в пользу любой другой стратегии Р, так как его средний выигрыш при этом не увеличится (при условии, что второй игрок сохранит выбранную им стратегию . Аналогично, второму игроку невыгодно отклоняться от своей стратегии $Q^*$ в пользу любой другой стратегии Q, так как средний выигрыш его противника при этом не уменьшится.

Стратегии $P^*,Q^*$, определяющие седловую точку, называются оптимальными стратегиями игроков. Они образуют равновесие по Нэшу в смешанном расширении игры.
\newpage

\textbf{Задания:}

1. Напишите формулу математического ожидания величины выигрыша первого игрока в развернутом виде и докажите, что матричное представление (18) дает тот же результат.

\textit{Решение:}

$a_{ij}$ - элементы матричной игры.

При использовании смешанных стратегий выигрыш игрока 1 оказывается случайной величиной с распределением, порожденным смешанными стратегиями на множестве всех ситуаций игры. Так как игроки выбирают строки и столбц независимо, то вероятноть случайной величины оказаться равной $a_{ij}$ равна:
$$P(\xi = a_{ij}) = p_i q_j, \quad i = (1, \ldots, m), \quad j = (1, \ldots, n)$$

Поэтому выигрыш игрока 1 в ситуации в смешанных стратегиях полагается равным математическому ожиданию выигрыша в чистых стратегиях:
$$\mathbb{E}(P,Q) = \sum_{i=1}^{m} \sum_{j=1}^{n} a_{ij} p_i q_j = PAQ^T$$

Заметим, что игрок $2$ получит $-\mathbb{E}(P,Q)$. $\blacksquare$

2. Проверьте, что формула (18) написана правильно по правилам матричного умножения.

\textit{Решение:}

Формула (18) верна, так как мы имеем дело с матрицами типа:
$$ P = (p_1,p_2,\ldots,p_m): (1 \times m)$$
$$ A: (m \times n)$$
$$Q = (q_1,q_2,\ldots, q_n): (1 \times n)$$

Тогда:
$$ \mathbb{E} = PAQ^T = (1 \times m) \cdot  (m \times n) \cdot (n \times 1) = (1 \times n) \cdot (n \times 1) =  (1 \times 1)$$

Получили число, следовательно, формула верна. $\blacksquare$.

3. В неравенствах (19) участвуют любые смешанные стратегии Р и Q. Докажите, что если эти неравенства выполнены на всех чистых стратегиях Р и Q, то они будут выполнены и на всех смешанных стратегиях. 

\textit{Решение:}

Доказательство:

Необходимо доказать, что она является равновесной и для смешаннного расширения игры.
Пусть ситуация $i^*, j^*$ в чистых стратегиях является равновесной для матричной игры с матрицей $A = a_{ij}$. Тогда выполняются неравенства:
$$ a_{ij^*} \leq a_{i^*j^*} \leq h_{i^*j}, \quad  \forall i = 1,2, \ldots, m, \forall j = 1,2, \ldots, m \eqno (20)$$

Все чистые стратегии игрока являются ортами в $n$-мерном евклидовом пространстве и матрица является единичной. Следовательно, в смешанных стратегиях:
$$ P^* = e_{j^*}, Q^* = e_{i^*}$$

Тогда:
$$e_{j^*} A e_{i^*}^T = P^* A Q^* = a_{i^*j^*} $$
$$e_{j} A e_{i^*}^T = a_{i,j^*}$$
$$e_{j^*} A e_{i}^T = a_{i^*,j}$$

Таким образом неравенства (20) записываются в виде:
$$e_{j} A e_{i^*}^T  \leq e_{j^*} A e_{i^*}^T \leq e_{j^*} A e_{i}^T $$

где $e_{j^*} A e_{i^*}^T=a$ - какое-то число

Отсюда вытекает неравенство:
$$PAQ^{*T} \leq a \leq P^*AQ^T$$ 

для любых чистых стратегий, следовательно, т.е выполняется неравенство $$PAQ^{*T} \leq P^*AQ^{*T} \leq P^*AQ^T \eqno (19)$$

Тогда $(P^*,Q^*)$ - ситуация равновесия в смешанных стратегиях. $\blacksquare$
\newpage
\subsubsection{Формульное решение антагонистических игр}

Рассмотрим антагонистическую игру с матрицей $A: 2 \times 2$.
$$A = 
\begin{pmatrix}
  a_{11} & a_{12} \\  
  a_{21} & a_{22}  
\end{pmatrix}$$

Смешанная стратегия первого игрока есть вектор $P = (p,1-p)$. Он полностью определяется величиной $р$, лежащей на отрезке $[0;1]$. 

Аналогично, смешанная стратегия второго игрока есть вектор $Q = (q,1-q)$, полностью определяемый величиной $q$ из того же отрезка $[0,1]$.

Математическое ожидание $\mathbb{E}$ выигрыша первого игрока при выборе игроками своих смешанных стратегий Р и Q, то есть при выборе величин р и q, есть:
$$ \mathbb{E}(p,q) = PAQ^T = pqa_{11} + p \cdot (1-q)a_{12} + (1-p)qa_{21} + (1-p)(1-q)a_{22} \eqno (1)$$

Для того, чтобы стратегия Р была оптимальной, необходимо, чтобы она была не хуже каждой из двух чистых стратегий, то есть необходимо, чтобы выполнялись неравенства:
$$\left \{
\begin{matrix}
\mathbb{E}(p,q) \geq qa_{11} + (1-q)a_{12} \qquad (2) \\
\mathbb{E}(p,q) \geq qa_{21} + (1-q)a_{22} \qquad (3)
\end{matrix} \right .$$

Неравенство (2) и (3) преобразуются соответственно в:
$$\left \{
\begin{matrix}
(1-p)(qa_{11} + (1-q)a_{12} -qa_{21} - (1-q)a_{22}) \leq 0 \qquad (4) \\
p(qa_{11} + (1-q)a_{12} - qa_{21} - (1-q)a_{22}) \geq 0 \qquad (5)
\end{matrix} \right .$$

Если матрица $A$ не имеет седловой точки, то $0 < p < 1$. В этом случае обе части последних двух неравенств можно разделить на $1-p$ и на $p$ соответственно. В итоге получаем два неравенства, в совокупности эквивалетных равенству.
$$qa_{11} + (1-q)a_{12} - qa_{21} - (1-q)a_{22} = 0 \eqno (6)$$

Обозначим за $C = (a_{11} + a_{22}) - (a_{12} + a_{21}) \qquad (7)$. Тогда:
$$q^{*} = \frac {a_{22}-a_{12}}{C} \eqno (8)$$
$$1-q^{*} = \frac{a_{11} - a_{21}}{C} \eqno (9)$$

Отметим, что формулы (8), (9), задающие оптимальную стратегию второго игрока, мы получили, исходя из неравенств (2), (3), характеризующих оптимальное поведение первого игрока.

Написав аналогичные неравенства, характеризующих оптимальное поведение второго игрока, и проведя аналогичные рассуждения, получим компоненты оптимальной стратегии первого игрока:
$$p^{*} = \frac {a_{22}-a_{21}}{C} \eqno (10)$$
$$1-p^{*} = \frac{a_{11} - a_{12}}{C} \eqno (11)$$

Чтобы сосчитать цену игры – размер выигрыша первого игрока – следует подставить формулы (8) – (11) в (1). После простых преобразований получим:
$$ \mathbb{E}(p^*,q^*) = \frac{\Delta}{C} = \frac{a_{11}a_{22} - a_{12}a_{21}}{ (a_{11} + a_{22}) - (a_{12} + a_{21})} \eqno (12)$$

где $\Delta$ - определитель матрицы $A$.

\textbf{Вопросы и задания:}

\textbf{1}. Обоснуйте приведенное выше утверждение, что для того, чтобы стратегия Р была оптимальной, необходимо, чтобы выполнялись неравенства (2), (3).

\textit{Решение:}

Какой бы ни была матрица $A$, равновесные смешанные стратегии игроков существуют, или:
$$ \underset{P}{\max} \text{ } (\underset{Q}{\min} \text{ } PAQ^T) =  \underset{Q}{\min} \text{ } (\underset{P}{\max} \text{ } PAQ^T)$$

Общее значение минимаксов назовем значением матричной игры с матрицей выигрыша $A$. Обозначим за $V$.

Игроки 1,2 должны выбирать такие свои стратегии, которые в игре составляют седловую точку. 

\textit{Оптимальные стратегии} - равновесные стратегии игроков.

Значит определение седловой точки можно перезаписать как:
$$ PAQ^{*T} \geq V \geq P^*AQ^T$$

Теперь понятно почему неравенства (2),(3) верны - выбор игроком 1 оптимальной стратегии дает ему выигрыш не меньший, чем значение игры, что бы ни делал игрок 2. $\blacksquare$

\textbf{2}. На самом деле, неравенства (2), (3) являются не только необходимыми, но и достаточными условиями оптимальности стратегии Р. Докажите утверждение о достаточности этих условий.

\textit{Решение:}

Cледует из определения минимакса и предыдущего пункта. $\blacksquare$.

\textbf{3}. Выведите формулы (10), (11), определяющие оптимальную стратегию первого игрока.

\textit{Решение:}

Для того, чтобы стратегия Q была оптимальной, необходимо, чтобы она была не хуже каждой из двух чистых стратегий, то есть необходимо, чтобы выполнялись неравенства:
$$\left \{
\begin{matrix}
\mathbb{E}(p,q) \geq pa_{11} + (1-p)a_{21} \quad (2a)\\
\mathbb{E}(p,q) \geq pa_{12} + (1-p)a_{22} \quad (3a)
\end{matrix} \right .$$

Неравенство (2a) и (3a) преобразуются соответственно в:
$$\left \{
\begin{matrix}
(1-q)(a_{21} - a_{22}-a_{21}p+(a_{11}-a_{12}+a_{22})p) \leq 0 \qquad (4a) \\
q(a_{21} - a_{22}-a_{21}p+(a_{11}-a_{12}+a_{22})p) \geq 0 \qquad (5a)
\end{matrix} \right .$$

Если матрица $A$ не имеет седловой точки, то $0 < p < 1$. В этом случае обе части последних двух неравенств можно разделить на $1-q$ и на $q$ соответственно. В итоге получаем два неравенства, в совокупности эквивалетных равенству.
$$-a_{21} + a_{22} + a_{21}p - (a_{11}-a_{12}+a_{22})p = 0 \eqno (6a)$$

Обозначим за $C = (a_{11} + a_{22}) - (a_{12} + a_{21}) \qquad (7a)$. Тогда:
$$p^{*} = \frac {a_{22}-a_{21}}{C} \eqno (10)$$
$$1-p^{*} = \frac{a_{11} - a_{12}}{C} \eqno (11) \quad \blacksquare$$

4.  Выведите формулу (12), определяющую цену игры.

\textit{Решение:}
$$q^{*} = \frac {a_{22}-a_{12}}{C} \eqno (8)$$
$$1-q^{*} = \frac{a_{11} - a_{21}}{C} \eqno (9)$$
$$p^{*} = \frac {a_{22}-a_{21}}{C} \eqno (10)$$
$$1-p^{*} = \frac{a_{11} - a_{12}}{C} \eqno (11)$$
$$ \mathbb{E}(p^*,q^*) = PAQ^T = pqa_{11} + p \cdot (1-q)a_{12} + (1-p)qa_{21} + (1-p)(1-q)a_{22} \eqno (1) = $$
$$ = \frac {a_{22}-a_{12}}{C} \frac {a_{22}-a_{21}}{C} a_{11} + \frac {a_{22}-a_{21}}{C}\frac{a_{11} - a_{21}}{C}a_{12} +  \frac{a_{11} - a_{12}}{C}\frac {a_{22}-a_{12}}{C} a_{21}  + $$ 
$$+ \frac{a_{11} - a_{12}}{C}  \frac{a_{11} - a_{21}}{C}a_{22}$$
$$\mathbb{E}(p^*,q^*) = \frac{(a_{11} - a_{12} - a_{21} + a_{22}) (-a_{12}a_{21} + a_{11}a_{22})}{C^2} = \frac{C \cdot \Delta}{C^2} = \frac{\Delta}{C} (12) \quad \blacksquare$$

5.  Формулы (8) – (12) содержат в знаменателе величину С. Естественно, этими формулами нельзя пользоваться, если $C = 0$.

5.1 Докажите, что если матрица А не имеет седловой точки, то $C \neq 0$. 

\textit{Решение:}

Можно доказать следующее аналогичное утверждение:

Для того чтобы у матрицы А размера 2x2 существовала седловая точка достаточно, чтобы сумма элементов главной диагонали матрицы А равнялась сумме элементов её побочной диагонали:
$$a_{11} + a_{22} = a_{12} + a_{21}$$

\textit{Доказательство:}
Выразим в данном равенстве $a_{21}$:
$$a_{21} = a_{11} - a_{12} +a_{22}$$

Возможны только два случая: $a_{11} \leq a_{12}$ и $a_{11} > a_{12}$.

В первом случае, получаем, что $a_{21} < a_{22}$, что означает, что второй столбец содержит максимальный по солбцам, тогда существует оптимальная смешанная стратегия игрока 1, в которую чистая стратегия входит с нулевой вероятностью - стратегия игрока 1 - оптимальна. По определиню оптимальной стратегии, у матрицы $A$ существует седловая точка.

Аналогично доказывается второй случай. $\blacksquare$


5.2 Таким образом получается, что решение игры $2 \times 2$ состоит из двух этапов. На первом этапе проверяем, имеет ли матрица А седловую точку. Если да, то игра имеет решение в чистых стратегиях, и второй этап решения не нужен. Если нет, то переходим ко второму этапу, где по формулам (7) – (12) получаем решение в смешанных стратегиях.

6.  Согласно предыдущему заданию, если $С = 0$, то матрица А имеет седловую точку. Докажите, что обратное утверждение неверно.

\textit{Решение:}

Достаточно привести антипример:
$$
\begin{pmatrix}
1_{s} & 1_{s} \\
0 & 1 \\
\end{pmatrix}
$$

В данном примере в первой строке являются седловыми точками, $C = 1+1 -1 = 1 \neq 0 \quad \blacksquare$

7.  Рассмотрим игру в «Два пальца». Игроки одновременно показывают друг другу один или два пальца. Если число показанных пальцев одинаково, то выигрывает первый игрок, если разное, то второй. Размер выигрыша зависит от варианта игры. В первом варианте он равен 1 (или, соответственно, –1), независимо от числа показанных пальцев. Во втором варианте выигрыш равен сумме показанных пальцев (с соответствующим знаком). Напишите матрицу игры для первого и второго вариантов. Попробуйте предсказать оптимальные стратегии и средний выигрыш до проведения расчетов. Проверьте свои догадки расчетами.


\subsubsection{Графическая интерпретация решения игры $2 \times 2$}

8. Напишите на листе бумаги уравнения прямых для диаграммы первого игрока. Вычислите координаты точки пересечения этих прямых. Выпишите оптимальную стратегию первого игрока и размер его выигрыша.

\textit{Решение:}

$$A = 
\begin{pmatrix}
1 & 4 \\
3 & 2 \\
\end{pmatrix}
$$

Уравнение прямой первого игрока:
$$ x  = \frac{y-1}{3-1} \Rightarrow y = 2x+1$$
$$ x =\frac{y-4}{2-4} \Rightarrow y = 4-2x$$

Найдем точку пересечения:
$$2x+1 = 4-2x \Rightarrow x = \frac{3}{4}, y = 2.5$$

Т.е точка $N(0.75,0.25)$, откуда $p = 0.75, 1-p=0.25, y = 2.5$

9.  Проведите на бумаге аналогичные расчеты для второго игрока. Выпишите оптимальную стратегию второго игрока и размер выигрыша первого игрока (проигрыша второго).

\textit{Решение:}

Уравнение прямой второго игрока:
$$ x = \frac{y-1}{4-1} \Rightarrow y =3x +1  $$
$$ x = \frac{y-3}{2-3} \Rightarrow y = 3-x $$

Найдем точку пересечения:
$$3x +1 = 3-x \Rightarrow x = \frac{1}{2}, y = 2.5$$

Размер выигрыша равен: $$\frac{2-12}{3-7} = 2.5 = y $$

10. Чему соответствует ситуация, когда абсцисса точки пересечения находится за пределами отрезка [0, 1]? Каковы в этом случае оптимальные стратегии игроков?

\textit{Решение:}

Если у кого-то из игроков абсцисса точки пересечения не принадлежит отрезку (0,1), то игрок выбирает чистую стратегию: 1 игрок выбирает лучшую из худших точек, 2-ой игрок - наоборот.

11. Приведите пример матрицы, для которой один из игроков имеет бесконечно много решений в смешанных стратегиях, а другой имеет единственное решение. Как такая ситуация выглядит на диаграмме?

\textit{Решение:}

Такое возможно, когда по столбцам больше седловых точек, нежели по строкам. График в файле.

12. Приведите пример матрицы, для которой оба игрока имеют бесконечно много решений в смешанных стратегиях. Как такая ситуация выглядит на диаграмме?

\textit{Решение:}

Такое возможно, когда все элементы матрицы одинаковые. График - две горизонтальные прямые (в файле).

13. Проведите показанное на рис. 1 построение в Excel. Построенная конструкция должна быть универсальной, то есть давать решение задачи при любой матрице А (если оптимальных стратегий бесконечно много, то давать одну из них). Проверьте это на разных матрицах. В частности, полезно провести проверку на матрице с одинаковыми элементами.

\textit{Решение:}

Проведено в файле матрица.xlsx

\end{document}
