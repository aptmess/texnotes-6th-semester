\RequirePackage{ifluatex}
\let\ifluatex\relax

\documentclass[aps,%
12pt,%
final,%
oneside,
onecolumn,%
musixtex, %
superscriptaddress,%
centertags]{article} %% 
\topmargin=-40pt
\textheight=650pt
\usepackage[english,russian]{babel}
\usepackage[utf8]{inputenc}
%всякие настройки по желанию%
\usepackage[colorlinks=true,linkcolor=black,unicode=true]{hyperref}
\usepackage{euscript}
\usepackage{supertabular}
\usepackage[pdftex]{graphicx}
\usepackage{amsthm,amssymb, amsmath}
\usepackage{textcomp}
\usepackage[noend]{algorithmic}
\usepackage[ruled]{algorithm}
\usepackage{lipsum}
\usepackage{indentfirst}
\usepackage{babel}
\usepackage{pgfplots}
\pgfplotsset{compat=1.9}

\pgfplotsset{model/.style = {blue, samples = 100}}
\pgfplotsset{experiment/.style = {red}}
\selectlanguage{russian}

\setlength{\parindent}{2.4em}
\setlength{\parskip}{0.1em}
%\renewcommand{\baselinestretch}{2.0}

\usepackage{xcolor}
\usepackage{hyperref}
 
 % Цвета для гиперссылок
%\definecolor{linkcolor}{HTML}{799B03} % цвет ссылок
%\definecolor{urlcolor}{HTML}{799B03} % цвет гиперссылок
 
%\hypersetup{pdfstartview=FitH,  linkcolor=linkcolor,urlcolor=urlcolor, colorlinks=true}

\begin{document}

\begin{titlepage} 
\begin{center}
% Upper part of the page
%\textbf{\Large САНКТ-ПЕТЕРБУРГСКИЙ ГОСУДАРСТВЕННЫЙ ЭКОНОМИЧЕСКИЙ УНИВЕРСИТЕТ} \\[1.0cm]
%\textbf{\large Кафедра Прикладной Математики и Информатики}\\[3.5cm]
 
% Title
\textbf{}\\[10.0cm]
\textbf{\LARGE Эконометрика}\\[0.5cm]
\textbf{\Large ПМ-1701} \\[0.1cm]

%supervisor
\begin{center} \large
{Преподаватель:} \\[0.5cm]
\textsc {Курышева Светлана Владимировна }\\
%{viktor\_chernov@mail.ru}\\
\end{center}
% \begin{flushright} \large
%\emph{Рецензент:} \\
%д.ф. - м.н., профессор \textsc{Надеемся Нам Помогут}
%\end{flushright}
%\begin{flushright} \large
%\emph{Заведующий кафедрой:} \\
%д.ф. - м.н., профессор \textsc{Не Обмани Себя}
%\end{flushright}
\vfill 

% Bottom of the page
{\large {Санкт-Петербург}} \par
{\large {2020 г., 6 семестр}}
\end{center} 
\end{titlepage}

% Table of contents
\begin{thebibliography}{3}
\bibitem{eliseeva}
Эконометрика: Учебник/И.И.Елисеева и др.-М.:Проспект, 2009
\bibitem{praktikuma}
Практикум по эконометрике: Учебное пособие/И.И.Елисеева и др.,М.:Финансы и статистика,2006 
\bibitem{dop1}
Эконометрика: Учебник/В. С.Мхитарян и др.-М.:2008
\bibitem{dop2}
Доугерти К. Введение в эконометрику: Учебник. 2-е изд. / Пер. с англ. – М.: ИНФРА – М, 2007
\bibitem{dop3}
Берндт Э. Практика эконометрики: классика и современность. М.,2005
\end{thebibliography}
\tableofcontents
\newpage
\section{Парная регрессия}

\subsection{21.02.2020}

Дана зависимость спроса от цены:
$$ X = (15,14,12,11,9,10) $$
$$ Y = (5,4,6,10,18,10) $$

\textbf{1.} Необходимо построить поле корреляции и выбрать математическую функцию.

\begin{center}
	\begin{tikzpicture}
		\begin{axis}[xmin = 0, xmax = 20,ymin = 0, ymax = 20, grid = major,scale = 1.5,legend pos = north west]
		\legend{ 
	Начальные данные
	};
	\addplot[scatter,only marks] coordinates { (15,5) (14,4) (12,6) (11,10) (9,18) (10,10)} ;
		\end{axis}
	\end{tikzpicture}
\end{center}

По данной информации лучшей аппроксимации является нелинейная регрессия - степенная функция.

\textbf{2.} Найти линейное уравнение, используя МНК.
$$y = a + bx$$
Согласно формуле (7) получаем следующую систему уравнений:
$$ \left\{
\begin{matrix}
53 = 6a + 71b \\
575 = 71a + 867b \\
\end{matrix} \right. $$
Из данной системы уравнений находим значения параметров регрессии $a$ и $b$:
$$ a = 31.8385; b = -1.9441$$

Построим график прямой $$\widehat{Y} = 31.8385 -1.9441X$$
\begin{center}
	\begin{tikzpicture}
		\begin{axis}[xmin=0,xmax=20, grid = major,scale = 1.5,domain = 0:25]
		\legend{ 
	Уравнение регрессии, 
	Начальные данные
	};
	\addplot[dashed, model]{31.8385 -1.9441*x};
	\addplot[scatter,only marks] coordinates { (15,5) (14,4) (12,6) (11,10) (9,18) (10,10)} ;
		\end{axis}
	\end{tikzpicture}
\end{center}

Линейный коэффициент корреляции по формуле (10):
$$ r = -0.87378$$

\textbf{3.} Построить таблицу дисперсионного анализа:

\label{first_table_analiz}
\begin{table}[H]
	\begin{center}
		\begin{tabular}[t]{|c|c|c|c|c|} \hline
		Источник вариации & df & $SS$ & $MS$ & F-критерий\\ \hline
		Регрессия (r) & 1 & 101.417 & 101.417 & 12.9127 \\ \hline
		Остаток (e)& 4 & 31.4161 & 7.85404 & 1 \\ \hline
		Итого (t)& 5 & 132.833 & 26.5667 & x \\ \hline
		\end{tabular}
	\caption{Таблица дисперсионного анализа для примера}
	\end{center}
\end{table}

Найдем табличное значение распределения Фишера-Снедекора при заданном уровне значимости:
$$ F_{1-\alpha}(m,n-1-m) = F_{0.95}(1,4) = 7.71$$

\textbf{4.} Найти линейное уравнение регрессии, используя программу Excel.

\subsubsection{Hometask на 28.02.2020}

\textbf{5.} Дать интервальный прогноз спроса, для $x_p=9$

Выражение для \textbf{стандартной ошибки предсказываемого по линии регрессии значения} $\widehat{y}$:
$$ m_{\widehat{y_x}} = \sqrt{MS_E} \sqrt{\frac{1}{n}+\frac{(x_k-\overline{X})^2}{\sum(X-\overline{X})^2}} \eqno (37) $$

Для прогнозируемого значения $\widehat{y}$ доверительный интервал выглядит следующим бразом:
$$ \widehat{y_{x_k}} \pm t_{1-\frac{\alpha}{2}} \cdot m_{\widehat{y_x}} $$
$$ \widehat{y_{x_k}} - t_{1-\frac{\alpha}{2}} \cdot m_{\widehat{y_x}} \leq \widehat{y_{x_k}} \leq \widehat{y_p} + t_{1-\frac{\alpha}{2}} \cdot m_{\widehat{y_x}}  \eqno (38)$$

\textbf{Средняя ошибка прогнозируемого индивидуального значения} составит:
$$ m_y = \sqrt{MS_E} \sqrt{1+\frac{1}{n}+\frac{(x_k-\overline{X})^2}{\sum(X-\overline{X})^2}} \eqno (39)$$

\textbf{Доверительный интервал для $y_p$ }- предсказываемого значения регрессии:
$$ \widehat{y_p} - t_{\alpha}m_y \leq y_p \leq \widehat{y_p} + t_{\alpha}m_y \eqno (40) $$

Вычислим стандартную ошибку предсказываемого по линии регрессии значения $\hat{Y}$ по формуле (37):
$$ m_{\widehat{y_x}} = \sqrt{7.85404} \sqrt{\frac{1}{6} + \frac{(x_k-11.8333)^2}{26.8333}} $$

Подставляя различные значения из выборки $X$ мы можем узнать ошибку предсказываемого значения. Минимальная ошибка будет при подстановке $x_k = \overline{X}=11.8333$:
$$ m_{y_{\overline{X}}} = \sqrt{11.8333} \sqrt{\frac{1}{6}} = 1.14412$$

Построим доверительный интервал для $\widehat{Y}$ при каком-то произвольном значении $x_k$, например $x_k = 9$. Воспользуемся формулой (38).

Сначала вычислим значение линейной регресии в точке $x_k=9$:
$$ \widehat{y_{9}} =  31.8385 -1.9441 \cdot 9 = 14.3416$$

Затем вычислим стандартную ошибку в точке $x_k=9$:
$$ m_{\widehat{y_9}} = \sqrt{7.85404} \sqrt{\frac{1}{6} + \frac{(9-11.8333)^2}{26.8333}} = 1.91278$$

Теперь можно и построить доверительный интервал для уровня значимости $\alpha=0.05$:
$$ \widehat{y_{9}} - t_{0.975}\cdot m_{\widehat{y_9}} \leq \widehat{y_{9}} \leq \widehat{y_{9}} + t_{0.975}\cdot m_{\widehat{y_9}}$$
$$9.0309 \leq \widehat{y_{9}} \leq 19.6523$$

Средняя ошибка прогнозируемого индивидуального значения:
$$ m_y = \sqrt{MS_E} \sqrt{1+\frac{1}{n}+\frac{(x_k-\overline{X})^2}{\sum(X-\overline{X})^2}} \eqno  = \sqrt{7.85404} \sqrt{1+\frac{1}{6}+\frac{(9-11.8333)^2}{26.8333}} = 3.39304 $$

\textbf{Доверительный интервал для $y_p$ }- предсказываемого значения регрессии: 
$$ 4.92101 \leq \widehat{y_{9}} \leq 23.7622$$

\textbf{6.} Используя Excel найти уравнение регрессии по степенной функции.

\begin{center} 1. Степенная функция \end{center}

Модель:
$$ y = a \cdot x^b \cdot \varepsilon $$

Логарифмируем обе части равенства (линеаризация):
$$ \ln y =\ln a + b\ln x + \ln \varepsilon $$

Замена переменных:
$$ \ln y = z, \alpha_1 = \ln a, t = \ln x, \varepsilon_1 = \ln \varepsilon $$

Линейный вид:
$$ z =\alpha_1 + b \cdot t + \varepsilon_1 $$

В нашем случае нам нужно прологарифмировать наши ряды $x$ и $y$, найти коэффициенты $a$ и $b$ линейной функции и перейти обратно к степенной, сделав замену $a={\rm e}^{\alpha_1}$

$$ X = (15,14,12,11,9,10) $$
$$x_{new} = \ln X = (2.70805, 2.63906, 2.48491, 2.3979, 2.19722, 2.30259)$$
$$ Y = (5,4,6,10,18,10) $$
$$ y_{new} = \ln Y = (1.60944, 1.38629, 1.79176, 2.30259, 2.89037, 2.30259)$$
Находим коэффициенты методом МНК:

$$ \alpha_1 = 8.58854, b = -2.66456 $$

Исходная степенная модель имеет вид:

$$ y = {\rm e}^{\alpha_1}x^b = 5369.78 \cdot x^{-2.66456} $$

Построим график степенной функции $$y= 5369.78 \cdot x^{-2.66456}$$

\begin{center}
	\begin{tikzpicture}
		\begin{axis}[xmin=8,xmax=16, grid = major,scale = 1.5,domain = 8:16]
		\legend{ 
	Уравнение регрессии, 
	Начальные данные
	};
	\addplot[dashed, model]{5369.78*x^-2.66456};
	\addplot[scatter,only marks] coordinates { (15,5) (14,4) (12,6) (11,10) (9,18) (10,10)} ;
		\end{axis}
	\end{tikzpicture}
\end{center}

\subsection{28.02.2020}

\textbf{Задача 1.}

Изучается зависимость спроса от цены: $Y(X)$. Имеется выборка из 5 наблюдений и по данным наблюдениям построена модель линейной регресси: $Y = 12.48 - 1.05 X$. Линейный коффициент корреляции для данной модели равен $r_{xy} = -0.97206$. Дисперсия спроса $\sigma_Y^2 = 2.3336$, а дисперсия цены равна $\sigma_X^2 = 2$. Также известно, что $\bar{X} = 4,\bar{Y} = 8.28$

Найти уравнение регрессии: $X(Y)$.

\textit{Решение:}

Необходимо найти уравнение $X = A + BY$.

Известна формула $|r_{yx}| = \sqrt{b \cdot B}$, поэтому: $$ B = \frac{r_{yx}^2}{b} = -0.8999$$
$$ A = \overline{X} - B\overline{Y} = 4 + 0.8999 \cdot 8.28 = 11.4512$$

Соответственно, уравнение регресси принимает вид:
$$ X = 11.4512  -0.8999Y$$

\textbf{Задача 2.}

Коффициент регрессии $a=4$, его стандартная ошибка $m_a = 0.8$, коффициент регрессии $b=0.12$, его стандартная ошибка равна $m_b = 0.045$.Также даны результаты регрессии:
\begin{table}[H]
	\begin{center}
		\begin{tabular}[t]{|c|c|c|c|c|} \hline
		Источник вариации & df & $SS$ \\ \hline
		Регрессия (r) & 1 & 1400 \\ \hline
		Остаток (e)& 18 & 3600  \\ \hline
		Итого (t)& 19 & 5000  \\ \hline
		\end{tabular}
	\end{center}
\end{table}
\textit{Решение:}
\begin{enumerate}
	\item Найти общее количество наблюдений.$$df_{SS_E} = n - 1 - m \Rightarrow n =df_{SS_E} + 1 + 1 = 20$$
	\item Оценить значимость уравнения регрессии c помощью $F$-критерия и его параметров($\alpha = 0.05$).
	$$t_b = \frac{b}{m_b} = 2.66667 > t_{0.975}(20-2) = 2.10092 - \text{значим}$$
	$$t_a = \frac{a}{m_a} = 5 > t_{0.975}(20-2) = 2.10092 - \text{значим}$$
	$$F_{stat} = \frac{MS_R}{MS_E} = \frac{SS_R}{m} \cdot \frac{n-1-m}{SS_E} = \frac{1400}{1} \cdot \frac{18}{3600} = 7 >$$
	$$ >   F_{1-\alpha}(m,n-1-m) = F_{0.95}(1,18) = 4.41 - \text{связь существенна}$$
	\item Найти коффициент детерминации.
	$$ R^2 = \frac{SS_R}{SS_T} = \frac{1400}{5000} = 0.28$$
	\item Дать интервальную оценку для коэффициента регрессии.
	$$\delta_b = \pm t_{table} \cdot {m_b}$$
$$ b - t_{1 - \frac{\alpha}{2}}(n-2) \cdot {m_b} \leq b \leq b + t_{1 - \frac{\alpha}{2}}(n-2) \cdot {m_b} \eqno (35)$$
$$ 0.0254585 \leq b \leq 0.214541$$
\end{enumerate}

Домашнее задание: практикум, парная регрессия, задачи 2,8 (стр.32).

\newpage

\subsubsection{Домашнее задание на 06.03.2020}

\textbf{Задание 8 (практикум)}

Моделирование прибыли фирмы по уравнению $y=ab^x$ привело к следующим результатам:
$$ Y = (10,12,15,17,18,11,13,19) $$
$$ \widehat{Y} = (11, 11, 17, 15, 20, 11, 14, 16) $$

\textit{Решение:}

1. Определите ошибку аппроксимации:
$$ \text{MAPE} = \frac{\sum \left| \frac{Y_i - \widehat{Y_i}}{Y_i} \right | \cdot 100\%} {n} = 9.7503$$

2. Найдите показатель тесноты связи прибыли с исследуемым в модели фактором:
$$ R^2 = r^2 = \frac{\sigma_{y,obyasn}^2}{\sigma_{y,obch}^2} = \frac{SS_R}{SS_T} = 1 - \frac{SS_E}{SS_T} = 1 - \frac{\sum (Y - \widehat{Y})^2}{\sum (Y - \overline{Y})^2}  = 1 - \frac{24}{79.875} = 0.699531$$
$$r_{xy} = \sqrt{R^2} = 0.836379$$

Связь является сильной.

3. Рассчитайте $F$-критерий Фишера:
$$F = \frac {MS_R} {MS_E } = \frac{\frac {SS_R} {df_R}}{\frac {SS_E} {df_E}} = \frac{r_{yx}^2}{(1-r_{yx}^2)} \cdot \frac{n-1-m}{m} = \frac{0.699531}{1-0.699531} \cdot \frac{8-1-1}{1} = 13.9688$$ 
$$F > F_{table}(m,n-1-m) = F_{table}(1,6) = 5.99 \Rightarrow $$

связь между результатом и фактором существенна.

\textbf{Задание 2 (практикум)}

При изучении спроса на телевизоры марки $N$ по 19 торговым точкам аналитики компании $ABC$ выявили следующую зависимость:
$$\ln y = 10.5 - 0.8 \ln x + e $$

До проведения эксперимента администрация считала, что эластичность спроса по цене для телевизоров марки $N$ составляет $-0.9$. Подтвердились ли результаты исследований? $t_b = -4.0$

\textit{Решение:}

Построим доверительный интервал для коэффициента $b$:
$$ b = -0.8 \text{т.к степенная функция}$$
$$ b - t_{1 - \frac{\alpha}{2}}(n-2) \cdot {m_b} \leq b \leq b + t_{1 - \frac{\alpha}{2}}(n-2) \cdot {m_b} \eqno (35)$$
$$t_b = \frac{b}{m_b} \Rightarrow m_b = \frac{b}{t_b} = \frac{-0.8}{-4.0} = 0.2$$
$$t_{1 - \frac{\alpha}{2}}(n-2)  = t_{1 - \frac{\alpha}{2}}(19-2) =  2.10982$$
$$ -1.22196 \leq b \leq -0.378037$$

Так как $b$ для степенной функции является коэффициентом эластичности, то значение $-0.9$ поподает в доверительный интервал. Гипотеза об эластичности не отвергается.

\section{Множественная регрессия}

\subsection{06.03.2020}

\textbf{Задача 1}

Дана зависимость прибыли $y$ от $x_1$ - единиц продукции и $x_2$ - производительности фондов:
$$y = (11, 17, 15, 12, 13, 8, 10, 14, 16, 18)$$
$$x_1 =(9, 15, 13, 10, 11, 7, 8, 13, 15, 20)$$
$$x_2 = (12, 20, 21, 13, 10, 6, 7, 19, 22, 18)$$

\textit{Решение:}

1. Найти уравнение регрессии:
$$\left \{
\begin{matrix}
	\sum y = n \cdot a + b_1 \sum x_1 + b_2 \sum x_2 \\[0.3cm]
	\sum y \cdot x_1 = a \sum x_1 + b_1 \sum x_1^2 + b_2 \sum x_1x_2\\[0.3cm]
	\sum y \cdot x_2 = a\sum x_2 + b_1 \sum x_1x_2 + b_2 \cdot \sum x_2^2
\end{matrix}
\right.
$$

Подставляя данные значения и решая систему, получим следующий результат:
$$\widehat{y} = 3.86985 + 0.57921x_1 + 0.170386x_2$$
$$ \widehat{y}  = 11.1274, 15.9657, 14.9777, 11.877, 11.945, 8.94663, 9.69623, 14.6369, \
16.3065, 18.521$$

2. Найти $R^2$ - коэффициент детерминации:

$$ SS_E  =3.97401, SS_R = 88.426, SS_T =92.4 $$
$$ R^2 = 1- \frac{SS_E}{SS_R} = 0.956991$$

3. Построим коррелляционную матрицу:
$$r_{y,x_1} = 0.958612,r_{y,x_2} = 0.874451,r_{x_1,x_2} = 0.786536$$
\begin{table}[H]
	\begin{center}
		\begin{tabular}[t]{|c|c|c|c|c|c|} \hline
		$\times$ & y & $x_1$ & $x_2$ \\ \hline
		y & 1 & 0.958612 & 0.874451 \\ \hline
		$x_1$ & 0.958612 & 1 & 0.786536  \\ \hline
		$x_2$ & 0.874451 & 0.786536 & 1  \\ \hline
		\end{tabular}
	\end{center}
\end{table}

4. Высчитывать коэффициент детерминации по корреляционной матрице:

Коэффициент детерминации может быть определен через матрицу парных коэффициентов корреляции:
$$ R^2 = 1 -\frac{| \Delta r |}{\Delta r_{11}}  = 0.956991$$
$$ \Delta r_{11} = 
\begin{vmatrix}
	1 & 0.786536 \\
	0.786536 & 1 \\
\end{vmatrix}$$

5. Нормированный коэффициент множественной корреляции:
$$ \hat{R^2} = 1 -(1-R^2) \frac{n-1}{n-m-1} = 0.944703 $$

6. Множественный коэффициент детерминации:
$$ r_{xy} = \sqrt {R^2} = 0.978259$$

7. Показатель частной корреляции:

При двухфакторной модели:
$$ \hat{y} = a +b_1x_1 + b_2x_2$$

Частная корреляция равна:
$$ r_{yx_1x_2} = \sqrt{1 - \frac{1 - R_{yx_1x_2}^2}{1-r_{yx_2}^2}} = 0.904016$$
$$ r_{yx_2x_1} = \sqrt{1 - \frac{1 - R_{yx_1x_2}^2}{1-r_{yx_1}^2}} = 0.68516$$
$$ R_{yx_1x_2}^2 = R^2 $$

8.Оценка значимости модели множественной регрессии:

$$F_{x_1} = \frac{R_{yx_1x_2}^2 - r_{yx_2}^2}{1-R_{yx_1x_2}^2} \cdot (n-3) = 31.3026$$
$$F_{x_2} = \frac{R_{yx_1x_2}^2 - r_{yx_1}^2}{1-R_{yx_1x_2}^2} \cdot (n-3) = 6.19373$$

Оба значения больше $F(m,n-m-1) = F(1,7) = 5.59$, поэтому включение в модель фактора $x_1$ после фактора $x_2$ статистически оправдано. 

9. Дать интервальный прогноз предполагая, что $x_1 = 16, x_2 = 21$

9.1 Определим точечный прогноз для данных:

Стандартная ошибка регрессии:
$$S =\sqrt{\frac{\sum \xi_i^2}{n-m-1}} = 0.753469$$

Ошибка прогнозного значения функции регрессии получим по формуле:
$$m_{\hat{y}_p}  = \sqrt{\frac{\sum \xi_i^2}{n-m-1} (X_p^T(X^T X)^{-1}X_p)} = S  \sqrt {X_p^T(X^T X)^{-1}X_p}$$

где:
$$X_p = 
\begin{pmatrix}
	1  \\
	16 \\
	21 
\end{pmatrix}$$
$$X_p^{T} = 
\begin{pmatrix}
	1 & 16 & 21
\end{pmatrix}$$

Матрица $X^TX$ составляется из коэффициентов в правой части при поиске уравнения регрессии:
$$\left \{
\begin{matrix}
	\sum y = n \cdot a + b_1 \sum x_1 + b_2 \sum x_2 \\[0.3cm]
	\sum y \cdot x_1 = a \sum x_1 + b_1 \sum x_1^2 + b_2 \sum x_1x_2\\[0.3cm]
	\sum y \cdot x_2 = a\sum x_2 + b_1 \sum x_1x_2 + b_2 \cdot \sum x_2^2
\end{matrix}
\right.
$$
$$\left \{
\begin{matrix}
	n & \sum x_1 & \sum x_2 \\[0.3cm]
	\sum x_1 & \sum x_1^2 & \sum x_1x_2 \\[0.3cm]
	\sum x_2 & \sum x_1x_2 & \sum x_2^2
\end{matrix}
\right \}$$

Высчитывая ошибка прогнозного значения функции регрессии, получим:
$$m_{\hat{y}_p} = 0.361065$$

Предсказанное значение:
$$\hat{y} (16,21)= 16.7153$$

$97.5\%$ квантиль распределения Стьюдента с $n-m-1 = 10 - 2 - 1 = 7$ степенями свободы равен $t_{97.5\%} (7) = 2.36462$

Предельная ошибка:
$$\delta_{\hat{y}_p} = t_{table} m_{\hat{y}_p} = 0.853782$$

Интервальная оценка прогнозного значения функции регрессии определяется по формуле:
$$\hat{y} - \delta_{\hat{y}_p} \leq \hat{y_{gen}} \leq \hat{y} + \delta_{\hat{y}_p}$$
$$15.8615 \leq \hat{y_{gen}} \leq 17.5691$$

9.2 Доверительный интервал для индивидуального прогнозного значения:
$$ \hat{y} - \delta_{\hat{y}_p} \leq \hat{y_{pred}} \leq \hat{y} + \delta_{\hat{y}_p}$$
$$m_{y_i}  = \sqrt{\frac{\sum \xi_i^2}{n-m-1} (1 + X_p^T(X^T X)^{-1}X_p)} = S  \sqrt {1 +X_p^T(X^T X)^{-1}X_p}$$

Путем подстановки значений, получаем:
$$14.7396 \leq \hat{y}_{pred} \leq 18.691$$

Домашнее задание: построить по этим же данным степенную модель и дать прогноз для тех же значений.

\subsection{Домашнее задание на 13.03.2020}

Дана зависимость прибыли $y$ от $x_1$ - единиц продукции и $x_2$ - производительности фондов:
$$y = (11, 17, 15, 12, 13, 8, 10, 14, 16, 18)$$
$$x_1 =(9, 15, 13, 10, 11, 7, 8, 13, 15, 20)$$
$$x_2 = (12, 20, 21, 13, 10, 6, 7, 19, 22, 18)$$

Построить степенную модель и дать прогноз для тех же значений:

\textit{Решение:}

1. Степенная модель выглядит следующим образом:

Модель:
$$ y = a \cdot x_1^b \cdot x_2^c\cdot \varepsilon $$

Логарифмируем обе части равенства (линеаризация):
$$ \ln y =\ln a + b\ln x_1 + c\ln x_2 + \ln \varepsilon $$

Получаем уравнение, решая методом МНК:
$$ \ln y = 0.750122 + 0.592206\ln x_1 + 0.141373\ln x_2 + \ln \varepsilon $$

В степенном виде это выглядит, как:
$$y = e^{0.750122} + x_1^{0.592206} + x_2^{ 0.141373}$$

В данной моделе можно сравнивать коэффициенты эластичности между собой, так как степенная функция.

2. Доверительный интервал для прогнозного значения:

Стандартная ошибка:
$$ S = 0.053857$$

Прогнозное значение по модели:
$$\hat{y}(x_1 = 16,x_2 = 21) = e^{0.750122} + x_1^{0.592206} + x_2^{ 0.141373} = 16.8185 $$
$$\log{\hat{y}} = 2.82248$$

Высчитываем среднюю ошибку прогноза. Берем вектор $X_p = (1,\ln 16, \ln 21)$ и матрицу для регрессии, приведенной из степенноого вида к линейному.
$$m_{y_i} =  0.0593691$$

Предельная ошибка:
$$\delta_{\hat{y}_p} = t_{table}(n-1-m) m_{y_i} =2.36462 \cdot 0.0593691 = 0.140386 $$

Прогноз для логарифмов:
$$ \log{\hat{y}} -  \delta_{\hat{y}_p}  \leq \hat{\ln y}_{pred} \leq \log{\hat{y}} +  \delta_{\hat{y}_p}$$
$$2.68209 \leq \hat{\ln y}_{pred} \leq  2.96286$$

Прогноз для степенной функции (переходим от логаримов:)
$$ 14.6157 \leq \hat{y}_{pred} \leq 19.3533$$

\subsection{13.03.2020}

Продолжаем работать с данными из задачи, рассмотренной на предыдущих уроках.

10. Оценить наличие или отсутствие гетероскедастичности:

Тесты и критерии гетероскедастичности: 

1. Иногда график позволяет предположить отсутствие гетероскедастичности.

2. Тест ранговой корреляции Спирмэна. Остатки рассматриваются по модулю, устанавливается их зависимость от значений фактора $x$. Если $|e_i|$ и $x_i$ будут коррелированы, то делается вывод о гетероскедатичности.
$$ R = 1 - \frac{6\sum d^2}{n \cdot (n^2 - 1)}$$
$$t_R < t_{\alpha}$$

Проведем данный тест для наших данных. Используем модель множественной линейной регрессии, найденную на занятии 06.03.2020:

$$y = (11, 17, 15, 12, 13, 8, 10, 14, 16, 18)$$
$$x_1 =(9, 15, 13, 10, 11, 7, 8, 13, 15, 20)$$
$$x_2 = (12, 20, 21, 13, 10, 6, 7, 19, 22, 18)$$

$$\widehat{y} = 3.86985 + 0.57921x_1 + 0.170386x_2$$
$$ \widehat{y}  = 11.1274, 15.9657, 14.9777, 11.877, 11.945, 8.94663, 9.69623, 14.6369, \
16.3065, 18.521$$

Остатки модели:
$$y - \widehat{y}  = (-0.127369, 1.03429, 0.0223198, 0.123035, 1.05498, $$
$$-0.946634, 0.30377, -0.636909, -0.306486, -0.520994)$$
Возьмем модуль остатков:
$$ |y - \widehat{y}| = (0.127369, 1.03429, 0.0223198, 0.123035, 1.05498, 0.946634,$$ 
$$,0.30377,0.636909, 0.306486, 0.520994)$$

Перейдем к рангам в данном ряду и в рядах $x_1 ,x_2$:
$$r_{res} =  (8, 2, 10, 9, 1, 3, 7, 4, 6, 5)$$
$$r_{x_1} = (8, 2.5, 4.5, 7, 6, 10, 9, 4.5, 2.5, 1)$$
$$r_{x_2} = (7, 3, 2, 6, 8, 10, 9, 4, 1, 5)$$

Посчитаем квадрат разности для рангов остатков и рангов временных рядов и посчитаем статистику Кэндалла для проверки гетероскедастичности $x_1$:
$$ \rho = 1 - \frac{6\sum d^2}{n \cdot (n^2 - 1)} = 1 - \frac{6 \cdot 141 }{10 \cdot 99} =0.145455 $$

Статистика $t_R$ высчитывается следующим образом:
$$t_R = \frac{\rho}{\sqrt{\frac{1-\rho^2}{n-1-m}}} =0.41583 $$
$$t_{1-\frac{\alpha}{2}}(n-1-m) = t_{97.5\%}(7) = 2.306$$
$$t_R < t_{1-\frac{\alpha}{2}}(n-1-m) $$

Следовательно, остатки не обладают гетероскедастичностью, а обладают гомоскедастичностью.
Для $x_2$ считается аналогично:
$$t_R = -0.650828 <  t_{1-\frac{\alpha}{2}}(n-1-m)$$

Остатки $x_2$ также не обладают гетероскедастичностью.

3. Тест Гольдфельда-Квандта

Разберем применение данного метода на основе задачи.

\textbf{Задача 2 (практикум стр.123)}

По 20 наблюдениям была построена модель зависимости расходов на питание:
$$\hat{y} = 20.84 + 0.44x, r_{yx}^2 = 0.916$$

Даны значения $x$ и остаков $e_i$:

Применим Тест Гольдфельда-Квандта. Для начала запишем в таблицу значения $x$, $e_i$, $\hat{y}$, полученное из уравнения регрессии и $y = \hat{y_i} + e_i$.

После этого исключим $n=4$ центральных наблюдений и разобьем совокупность наших данных на две части:

\begin{enumerate}
	\item Со значениями $x$, ниже центральных
	\item Со значениям $x$, выше центральных
\end{enumerate}

Заметим, что в исходных данных, значения $x$ должны быть расположены по возрастанию.
В каждой части осталось по 8 наблюдений.
Найдем уравнение регрессии по каждой части:
$$\hat{y} = 5.795 + 0.629927\cdot x, F_{stat_1} = 72.848 $$
$$\hat{y} = 38.6056 + 0.364033 \cdot x, F_{stat_2} = 26.2807$$

Находим сумму квадратов остатков для каждой группы:
$$sum_1 = 68.8251$$
$$sum_2 = 1650.31$$

Найдем соотношение:
$$R = \frac{sum_2}{sum_1} = 23.9784$$

Сравним эту величину с табличным значеним $F$-критерия: $$F(n-1-m,n-1-m) = F(8-1-1,8-1-1) = F(6,6) = 4.28$$
$$R >F(6,6)$$

Следовательно, делаем вывод о гетероскедастичности остатков.

\subsection{Домашнее задание на 20.03.2020}

4. Тест Парка:

Если остатки зависят от переменной, значит их дисперсия непостоянна, а значит они гетероскедастичны. Если есть гомоскедастичность - остатки ни от чего не зависят

Строится регрессия вида:
$$\ln e^2 = a + b\ln x$$

Уравнение регрессии для наших данных:
$$\ln e^2 = -0.887929 + 1.03412 \cdot \ln x$$

$$R^2 = 1 -\frac{SS_E}{SS_T}  = 0.163975$$
$$F_{stat} = 3.53046 < F_{m=1,n-1-m = 18} = 4.41$$

По этим данным можно сделать вывод, что наши остатки гомоскедастичны - $R^2$ не высок, регрессия в принципе не значима, поэтому остатки гомоскедастичны по тесту Парка.

5. Тест Глейзера:

В тесте Глейзера мы оцениваем значимость абсолютных значений остатков от значений фактора x в виде функции
$$|e| = a + b\cdot x^c$$

где в качестве $c$ задается какое-то значение степени.
По данным строится регрессия, далее высчитываются $R^2$ и $t-stat$ для параметров регрессии и на основании этих значений выдается вердикт о гетероскедастичности.

В нашем случае:

Для с=1:
$$\hat{y} = 4.09+ 0.04\cdot x$$
$$R^2 = 0.29, t_{stat:1,2} =  1.71,2.76$$

Видим, что качество модели плохое, а один из коэффициентов регрессии не значим.
Попробуем от $c=2$:
$$\hat{y} = 6.97 + 0.0001\cdot x$$
$$R^2 = 0.247916, t_{stat:1,2} = 3.95,2.43$$

Коэффициенты стали значимыми.

Полученная величина зависит от остатков - гетероскедастичны.


\end{document}

