\RequirePackage{ifluatex}
\let\ifluatex\relax

\documentclass[aps,%
12pt,%
final,%
oneside,
onecolumn,%
musixtex, %
superscriptaddress,%
centertags]{article} %% 
\topmargin=-40pt
\textheight=650pt
\usepackage[english,russian]{babel}
\usepackage[utf8]{inputenc}
%всякие настройки по желанию%
\usepackage[colorlinks=true,linkcolor=black,unicode=true]{hyperref}
\usepackage{euscript}
\usepackage{supertabular}
\usepackage[pdftex]{graphicx}
\usepackage{amsthm,amssymb, amsmath}
\usepackage{textcomp}
\usepackage[noend]{algorithmic}
\usepackage[ruled]{algorithm}
\usepackage{lipsum}
\usepackage{indentfirst}
\usepackage{babel}
\usepackage{pgfplots}
\usepackage{setspace}
\linespread{1.15}
\pgfplotsset{compat=1.9}

\pgfplotsset{model/.style = {blue, samples = 100}}
\pgfplotsset{experiment/.style = {red}}

\selectlanguage{russian}

\setlength{\parindent}{2.4em}
\setlength{\parskip}{0.1em}
%\renewcommand{\baselinestretch}{2.0}

\usepackage{xcolor}
\usepackage{hyperref}
 
 % Цвета для гиперссылок
%\definecolor{linkcolor}{HTML}{799B03} % цвет ссылок
%\definecolor{urlcolor}{HTML}{799B03} % цвет гиперссылок
 
%\hypersetup{pdfstartview=FitH,  linkcolor=linkcolor,urlcolor=urlcolor, colorlinks=true}

\begin{document}

\begin{titlepage} 
\begin{center}
% Upper part of the page
%\textbf{\Large САНКТ-ПЕТЕРБУРГСКИЙ ГОСУДАРСТВЕННЫЙ ЭКОНОМИЧЕСКИЙ УНИВЕРСИТЕТ} \\[1.0cm]
%\textbf{\large Кафедра Прикладной Математики и Информатики}\\[3.5cm]
 
% Title
\textbf{}\\[10.0cm]
\textbf{\LARGE Теория и системы поддержки принятия решения}\\[0.5cm]
\textbf{\Large ПМ-1701} \\[0.1cm]

%supervisor
\begin{center} \large
{Преподаватель:} \\[0.5cm]
\textsc {Фролькис Виктор Абрамович}\\
% {viktor\_chernov@mail.ru}\\
\end{center}
% \begin{flushright} \large
%\emph{Рецензент:} \\
%д.ф. - м.н., профессор \textsc{Надеемся Нам Помогут}
%\end{flushright}
%\begin{flushright} \large
%\emph{Заведующий кафедрой:} \\
%д.ф. - м.н., профессор \textsc{Не Обмани Себя}
%\end{flushright}
\vfill 

% Bottom of the page
{\large {Санкт-Петербург}} \par
{\large {2020 г., 6 семестр}}
\end{center} 
\end{titlepage}

% Table of contents
\begin{thebibliography}{3}
\bibitem{rob}
Теория игр
\end{thebibliography}
\tableofcontents
\newpage

\section{11.02.2020}
\subsection{Критерий}
\textbf{Опр:} \textit{Критерий оптимальности} - совместно сформулированная цель в задаче оптимиации. Слово цель неоднозначно - оно имеет разный смысл: цель намерения и цель планов.
Изучая данную дисциплину, мы будем формулировать критерий в смысле \textit{цели плана}. 

Если мы говорим об оптимальности,то мы должны понимать, то речь идет о целе, которую мы измеряем в некоторой шкале.
Следовательно, цель должна быть измеряемой в рамках некоторой шкалы.

Математически мы можем описать данную шкалу как кортеж: 
$$<X,Y,f>$$

где $X$ - множество объектов, которые мы измеряем, $Y$ - знаковая система, а $f: X \stackrel{f}{\to} Y$ - отображение.

Элементы $y \in Y$ называются делениями шкалы. $Y$  -область значений функции $f$, $X$ - область определения функция $f$, $f$ -измерение.

Шкалы принято классифицировать по \textit{типам} измеряемых данных.

\textbf{Пример:} имеется три персоны: Иванов, Петров, Сидиров. $Y$ - измеряется в сантиметрах, с помощью которой мы можем оценить Иванова, Петрова и Сидирова. Знаковая система тогда будет являться следующим списком $(170,180,160)$. Мы можем поставить их в различных порядках. Это зависит от выбора \textit{цели} (рост, вес и.т.д).

В общем случае для оценивания объектов $X$ потребется $n$ признаков:
$$ x_o \in X: p_1,p_2,...,p_N$$

Каждый признак оценивается по своей шкале:
$$ <X, Y_j,f_j >  $$

где $j$ - номер наблюдения.

Для каждого объекта мы можем получить какое-то минимальное и максимальное значение по признаку:
$$\left[ y_{j,min},y_{j,max}\right ] $$

Задача многокритериального выбора обычно не ограничиваются поиском лучшего объекта - каждому из рассматриваемых объектов присваивается определенный рейтинг. Так как выбор достигается на лучшем объекте, то остальные объекты оцениваются по степени близости к лучшему объекту.
Меры достижения цели. 

Граница шкал является идеальной целью, которая обычно является недостижимой, поэтому задача многокрителиального выбора основана на компромиссе.

Реальные цены соответствут некоторым граничным значениям. В рамках реальной цели индивидум будет пытаться достичь какую-то точку внутри идеальной цели. В рамках реальной цели будем стремиться к верхней границе реальной цены - наилучшая реальная цена.

В рамках неравенств это выглядит следующим образом:
$$ y_{j,min} \leq y_{j,n} \leq c_j \leq y_{j,b} \leq y_{j,max}$$

Реальная цель используется в качестве базиса. Для сравнения можно использовать предикат бесконечнозначной логики:
$$ P \geq (f_j{(x_i)},c_{j}) $$

где $f_j{(x_i)}$ - оценка, а $c_j$ - критерий(база сравнений)

\textbf{Опр}: Логистическая функция, заданная на множестве утверждений и принимающая какое-то значение на основании утверждения, называется \textit{предикатом} $P$. В нашем случае наша функция может принимать значение на каком-то интервале.

\textbf{Опр}: \textit{k-значимая логика} - функция принимает $k$ значений.

\textbf{Опр}: \textit{Бесконечнозначимая логика} - функция принимает значения от $(0;1)$

\textbf{Опр}: \textit{Двухзначная логика} - принимает значения "да" или "нет"

У нас может допускаться частичное достижение цели (с какой-то определенной большой вероятностью), так как мы не всегда сможем достичь значение цели по всевозможным параметрам системы.

\textbf{Замечание:} можно ввести предикаты для объектов.
\newpage

\textbf{Опр:} \textit{Отношение предпочтения} объектов удовлетворяет двум условиям:

1. Если $(A,B) \Leftrightarrow f(A) > f(B) \vee (B,A) \Leftrightarrow f(B) > f(A) \vee (B \equiv A) \Leftrightarrow f(B) = f(A) $

2. Если $(A,B) > (B,C) \Rightarrow (A,B) > (A,C) $

Событие $(A,B)$ читается как "A предпочитетельнее B"

\subsection{Измерения и шкалы}

Критерий - цель, измеренная в некоторой шкале. Измерение - некоторая итерация, по которой мы наблюдаем состояние объекта, и соответствующее ей некоторое число.

Различным состоянием ставятся разные символами, а неразличимым состояниям - одинаковые символы. Для установления тождества различия состояния должны выполняться 3 аксиомы:

1. $ A = B \text{ | } A \neq B $

2. $ A = B \Leftrightarrow B=A $

3. $ A=B, B=C \Rightarrow A = C$ 

\subsubsection{Типы шкал}
\begin{center}
\textbf{1. Шкала наименований (номинальная,классификационная).}
\end{center}

По данной шкале объекты либо совпадают, либо различаются. Т.е каждому объекту сопоставляется уникальное имя, позволяющее отличать один объект от другого. Если имена совпадают, то объекты тождественны, иначе объекты различаются.

Способы обработки объектов в номинальной шкале:
посчитать количество встречающихся элементов, высчитываются относительные частоты, сравнение частот между собой, мода объекта.
\begin{center}
\textbf{2. Порядковая шкала (ранговая) }
\end{center}

Такая шкала позволяет реализовать предпочтение одного объекта по отношению к другому объекту.

2.1 \textit{ Шкала строгого порядка} - если между двумя любыми элементами можем установить отношение порядка. Для данной шаклы должно выполняться 3 аксиомы + выполняются следующие две аксиомы.
\begin{center}
4. $ A>B \text{ | } B>A$ 

5. $ A>B, B>C \Rightarrow A>C $
\end{center}

2.2 \textit{Шкала нестрогого порядка} - есть элементы, которые между собой мы упорядочить не можем. Для данной шкалы выполняются три аксиомы и еще две, представленные ниже:
\begin{center}
4а. $ A>B,B>A \Rightarrow A = B $

5a. $ \geq B, B \geq C \Rightarrow A \geq C $
\end{center}

2.3 \textit{Шкала линейного порядка} - если можно установить отношение предпочтения, если можно установить предпочтения между всеми парами объекта, т.е все объекты можно упорядочить.

2.4 \textit{Частичная шкала} - неполный класс.

2.5 \textit{Качественная шкала} - объекты разбиваются по классам.

2.6 \textit{Бальная шкала} - разным объектам разные ебаллы

\textbf{Опр:} Расстояние между объектами - насколько далеко находятся между собой объекты. Отклонение - расстояние от объекта до лучшего.

$$a_{ij} = 
\left\{
\begin{matrix}
1, x_i > x_j \\
0, x_i < x_j
\end{matrix} \right.
$$

Тогда, расстояние между двумя несоседними объектами определяется числом предпочтений:
$$ d_{ik} = \sum_{j=i}^{K-1} a_{jj+1} $$
$$ x_i \sim R_i , x_i \sim x_r: d_{ik} $$
$$ r_k = r_i +d_{ik} $$

2.7 \textit{Равномерная шкала} - равные интервалы сопостовляют равные отрезки шкал

2.8 \textit{Интервальная шкала}: $y = ax+b$, $a$ - масштаб, $b$ - смещение.

2.9 \textit{Новая шкала}: $y = ax+n\cdot b$, $a$ - масштаб, $b$ - смещение.

2.10 \textit{Шкала отношений}: $y = ax$ - позволяет сопоставлять величины, измеренные в разных шкалах. Для данной шкалы выполняются следующие свойства:

\begin{center}
1. $ A=P,B>0 \Rightarrow A + B > P $

2. $ A + B = B + A $

3. $ A = P, B = Q \Rightarrow A + B = P + Q $

4. $(A + B ) + C = A + (B + C) = A + B + C$
\end{center}

2.11 \textit{Абсолютная шкала }: переводит ответы в числа.

2.12 \textit{Полярная шкала }: оценивание противоположностей

Чем более слабая шкала, тем больше вероятность исказить результат измерения.

2.12 \textit{Мнение эксперта }: когда не удается произвести оценку эксперимента, эксперты принимают роль экспертного прибора.
\end{document}