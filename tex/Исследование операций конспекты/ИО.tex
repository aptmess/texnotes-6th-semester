\RequirePackage{ifluatex}
\let\ifluatex\relax

\documentclass[aps,%
12pt,%
final,%
oneside,
onecolumn,%
musixtex, %
superscriptaddress,%
centertags]{article} %% 
\topmargin=-40pt
\textheight=650pt
\usepackage[english,russian]{babel}
\usepackage[utf8]{inputenc}
%всякие настройки по желанию%
\usepackage[colorlinks=true,linkcolor=black,unicode=true]{hyperref}
\usepackage{euscript}
\usepackage{supertabular}
\usepackage[pdftex]{graphicx}
\usepackage{amsthm,amssymb, amsmath}
\usepackage{textcomp}
\usepackage[noend]{algorithmic}
\usepackage[ruled]{algorithm}
\usepackage{lipsum}
\usepackage{indentfirst}
\usepackage{babel}
\usepackage{pgfplots}
\usepackage{setspace}
\linespread{1.15}
\pgfplotsset{compat=1.9}

\pgfplotsset{model/.style = {blue, samples = 100}}
\pgfplotsset{experiment/.style = {red}}

\selectlanguage{russian}

\theoremstyle{plain}
\binoppenalty=10000
\newtheorem{theorem}{Theorem} %
\setlength{\parindent}{2.4em}
\setlength{\parskip}{0.1em}
\newtheorem{lemma}{Лемма}

\setlength{\parindent}{2.4em}
\setlength{\parskip}{0.1em}
%\renewcommand{\baselinestretch}{2.0}

\usepackage{xcolor}
\usepackage{hyperref}

\begin{document}

\begin{titlepage} 
\begin{center}
% Upper part of the page
%\textbf{\Large САНКТ-ПЕТЕРБУРГСКИЙ ГОСУДАРСТВЕННЫЙ ЭКОНОМИЧЕСКИЙ УНИВЕРСИТЕТ} \\[1.0cm]
%\textbf{\large Кафедра Прикладной Математики и Информатики}\\[3.5cm]
 
% Title
\textbf{}\\[10.0cm]
\textbf{\LARGE Исследование операций}\\[0.5cm]
\textbf{\Large ПМ-1701} \\[0.2cm]

%supervisor
\begin{center} \large
{Преподаватель:} \\[0.5cm]
\textsc {Чернов Виктор Петрович}\\
{viktor\_chernov@mail.ru}\\
\end{center}
% \begin{flushright} \large
%\emph{Рецензент:} \\
%д.ф. - м.н., профессор \textsc{Надеемся Нам Помогут}
%\end{flushright}
%\begin{flushright} \large
%\emph{Заведующий кафедрой:} \\
%д.ф. - м.н., профессор \textsc{Не Обмани Себя}
%\end{flushright}
\vfill 

% Bottom of the page
{\large {Санкт-Петербург}} \par
{\large {2020 г., 6 семестр}}
\end{center} 
\end{titlepage}

% Table of contents
\begin{thebibliography}{3}
\bibitem{Sulsky1994}
Sulsky D., Chen Z., Schreyer H. L.  A particle method for history-dependent materials // Computer Methods in Applied Mechanics and Engineering. --- 1994, V. 118. --- P. 179--196.
\bibitem{LiuLiu}
Liu G. R., Liu M. B. Smoothed particle hydrodynamics: a meshfree particle method. --- Singapore : World Scientific Publishing. --- 2003. --- 449 p.
\end{thebibliography}
\tableofcontents
\newpage
\section{Конспекты лекций}
\subsection{13.02.2020} 

\textbf{Отчет о результатах:} в каких пределах можно менять коэффициенты целевой функции чтобы оптимальынй план не изменился.

Перейдем к листу отчета об устойчивости.

\textbf{Теневая цена} - предельная полезность ресурса, компонент оптимального плана двойственной задачи, частная производная целелвой функции по правой части ограничения - величина, показывает на сколько единиц изменится результат, если изменить правую часть на единицу.

Представим задачу, меняем коэффициенты правой части, получили оптимальное решение $z^{*}$:
$$CX \to max$$
$$ \left\{
\begin{matrix}
AX \leq B \\ 
X \geq 0
\end{matrix}\right. $$
$$ Z^{*} = Z(B) = Z(b_1,b_2,...,b_n)$$
$$ \frac{\partial Z}{\partial b_i} =y_i^*$$
где $y_i^*$ - теневые цены, компоненты оптимального плана.

График предельной полезности является кусочно-линейным.

Отчет о пределах - сомнительная польза: если объем печенья будем равны $0$, то остается один бисквит.

\subsection{20.02.2020}
\subsubsection{Стратегии управления запасами и критерий оптимальности}

Рисуем типичный график зависимости запасов от времени. В начальный момент времени есть какой-то запас и он изменяется с течением времени. Склад является аккумулятором запасов потребителя. На склад, в свою очередь постсупает продукция поставщиков.

В какой-то момент времени запас склада пополняется на некоторую величину $V_1$.
Дефицит может отображаться двумя способами. 
\begin{itemize}
	\item Незадолженный дефицит - спустя какое-то время на склад при нулевом запасе приходит товар
	\item Задолженный дефицит - дефицит уходит в отрицательную область.
\end{itemize}

Последовательность пополнения запасов - результат принятия решений, она возникает тогда, когда потребительская система формирует заказ поставщикам.
$$ \left\{
\begin{matrix}
V_1 & V_2 & \ldots \\
t_1 & t_2 & \ldots
\end{matrix} \right.$$

Данный график носит название \textit{стратегии управления поставками}. Она состоит из отдельных управленческих решений. Какой график поставок лучше, т.е какая стратегия оптимальна?
В этом и состоит оптимизационная задача.

Сущестует три вида затрат:
\begin{itemize}
	\item Затраты связаны с поставками
	\item Затраты связаны с хранением
	\item Затраты связаны с дефицитом
\end{itemize}

Каждая из затрат подразделяется на постоянные и переменные затраты
Постоянные - не зависещее от объема. Затраты, связанные с поставкой, не зависят от объема: затраты на огранизацию.


Критерий оптмимальности: средние затраты в единицу времени были минимальными.

\subsubsection{Простейшие модели управления запасами. Формула Уилсона.}

Простешйая модель обладает тремя свойствами:

\begin{enumerate}
	\item Дефицит не допускается.
	\item Постоянный не меняющийся спрос, $\alpha$ -сколько единиц товара уходит на единицу времени
	\item Отсутствует неопределенность
\end{enumerate}

На графике мы заменяем кривые прямыми, угол наклона будет одинаковым по второму свойству.
Можно предположить, что поставка будет приходить точно в срок, и быть уверенным, что все так и будет.

Оптимальную стратегию следует искать среди графиков следующего вида:

\begin{center}
	\begin{tikzpicture}
		\begin{axis}[xmin=0,xmax = 16, grid = major,scale = 0.9,domain = 0:13]
		\legend{ 
	Начальные данные
	};
	\addplot[color=red,mark=x] coordinates { (0,5) (5,0) (5,3) (8,0) (8,4) (12,0)} ;
		\end{axis}
	\end{tikzpicture}
\end{center}

Обозначим за $a$ - постоянные затраты поставок. Постоянные затраты связанные с хранением мы устраняем из рассмотрения. Переменная составляющая по поставкам - тоже исключается, так как мы на нее не можем влиять - она изменяется от нас не зависяще. $b$ - коэффициент затрат по хранению - затраты по хранению товара на единицу времени. Размерность - количество единиц товара на единицу времени.
Дефицитные поставки все исключаем.

Коэффициент $b$ на графике - единичный квадрат.

Допустим у нас есть два треугольника. Общие затраты равны суммы двух затрат $T = T_1 + T_2$, $Q=\alpha \cdot T$
Тогда средние затраты равны площади этих двух треугольников, то есть:
$$mse =  \frac{2a + b(\frac{1}{2}Q_1T_1+\frac{1}{2}Q_2T_2)}{T}$$

Так как $Q=\alpha \cdot T$, то:
$$mse= \frac{2a + b(\frac{1}{2}\alpha T_1^2+\frac{1}{2} \alpha T_2^2)}{T}$$

Необходимо минимизировать следующее выражение:
$$2a + \frac{1}{2}b\alpha(T_1^2+(T-T_1)^2) \to \min$$

Возьмем производную:
$$f'(T_1) = b\alpha(T_1 - (T-T_1)) = b\alpha (-T + 2T_1)=0$$
$$T_1 = \frac{1}{2}T,T_2 = \frac{1}{2}T$$

Следовательно, оптимальные решения нужно искать среди периодической модели с одинаковыми треугольниками. Теперь задача состоит в том, чтобы найти длину партии $Q$ и $T$ - пероид.

Затраты на одном цикле управления запасами:
$$L_{sum} = a+\frac{1}{2}bQT = a+b\frac{1}{2}\alpha T^2$$

Такие формулы не позволятют сравнивать стратегии, следовательно нужно сравнить средни затраты, поэтому поделим на длину цикла:
$$ L = \frac{a+b\frac{1}{2}\alpha T^2}{T} = \frac{a}{T} + \frac{1}{2} \cdot b \cdot \alpha\cdot  T \to \min$$
$$L'(T) = -\frac{a}{T^2} + \frac{1}{2}b\alpha = 0$$
$$T^* = \sqrt{\frac{2a}{b\alpha}} - \min$$
$$Q^* = \sqrt {\frac{2a\alpha}{b}} - \min$$
$$ L = \frac{a}{\sqrt{\frac{2a}{b\alpha}}} + \frac{1}{2}b\alpha \sqrt{\frac{2a}{b\alpha}} = 
\sqrt{\frac{ab\alpha}{2}} + \sqrt{\frac{ab\alpha}{2}} = \sqrt{2ab\alpha}$$

Данные формулы называются \textit{Формулами Уилсона}.
Если рассмотреть зависимость двух величин $L$ от $T$, то графически мы ищем минимум зеленой прямой на графике:
\begin{center}
	\begin{tikzpicture}
		\begin{axis}[xmin=0,xmax = 5, grid = major,scale = 1.1,domain = 0:5]
	\addplot[color=red] {3/x} ;
	\addlegendentry{затраты по поставкам}
	\addplot[color=blue] {1/2*3*2*x} ;
	\addlegendentry{затраты по хранению}
	\addplot[color=green] {3/x + 1/2*3*2*x} ;
	\addlegendentry{средние затрыты L}
		\end{axis}
	\end{tikzpicture}
\end{center}

Необходимо выбрать прямоугольник заданной площади с минимальным периодом и данный прямоугольник является квадратом.

Философское правило: лучше перебрать, чем недобрать.

\subsubsection{Простейшая модель с допущением незадолженного дефицита.}

\begin{center}
	\begin{tikzpicture}
		\begin{axis}[xmin=0,xmax = 16, grid = major,scale = 0.9,domain = 0:13,title=Незадолженный дефицит]
		\legend{ 
	Начальные данные
	};
	\addplot[color=red,mark=x] coordinates { (0,4) (5,0) (8,0) (8,4) (12,0)} ;
		\end{axis}
	\end{tikzpicture}
\end{center}

Обозначим за $T_1$ недефицитный период $(0;4): T_1$ и $(4,8):T_2$ - период дефицтного периода. $g$ - штраф за отсутствие товара.
$$\alpha,a,b,g, Q = \alpha \cdot T_1$$

$$L = \frac{a+b\frac{1}{2}Q \cdot T_1+g\cdot T_2}{T_1 + T_2} \to \min$$
Лемма о неправильной суммы дробей:
\begin{lemma}
	$\frac{A_1}{B_1} \leq \frac{A_2}{B_2}$
\end{lemma}
\textit{Доказательство:}
$$\frac{A_1}{B_1} \leq \frac{A_1+A_2}{B_1+B_2} \leq \frac{A_2}{B_2}$$
$$A_1B_1 +A_2B_2 \leq A_1B_1 + A_2B_1 $$
$$\frac{A_1}{B_1} \leq \frac{A_2}{B_2} \qquad \blacksquare$$

$\sqrt{2a\alpha b} < g$ - дефифит не выгоден,
$\sqrt{2a\alpha b} > g$ - выгоден дефицит.

\subsubsection{Простешная модель с задолженным дефицитом}
$$ X = \alpha T_1, S = \alpha T_2, \alpha,a,b,g$$
$$ Q = \alpha T $$

$S $ - задолженный дефицит.
$$L = \frac{a+bT_1X\frac{1}{2} + g T_2 S \frac{1}{2}}{T_1 + T_2} \to \min$$
$$L = \frac{a+bT_1^2\alpha \frac{1}{2} + g T_2^2 \alpha \frac{1}{2}}{T_1 + T_2} \to \min$$

Приравниваем к нулю производные уравнений и решаем систему.

$$T_2 = \frac{b}{g}T_1$$
$$T_1^* = \sqrt{\frac{2a}{b\alpha \cdot (1+\frac{b}{g})}}$$
$$ T_2^* = \frac{b}{g}  \sqrt{\frac{2a}{b\alpha \cdot (1+\frac{b}{g})}} =  \sqrt{\frac{2agb^2}{b\alpha \cdot (g+b) g^2}} =\sqrt{\frac{2ab}{\alpha \cdot (g+b) g}} $$
В пределе:
$$T_1^* \to \sqrt{\frac{2a}{b\alpha }}$$
$$T_2^* \to 0$$
$$X*=\alpha T_1 = \alpha  \sqrt{\frac{2a}{b\alpha \cdot (1+\frac{b}{g})}} $$

Размер дефицита:
$$ S^* = \alpha T_2 = \alpha \sqrt{\frac{2ab}{\alpha \cdot (g+b) g}}$$

То есть при оптимальном случае, размер дефицита стремится к нулю, а $X \to Q$.
\end{document}