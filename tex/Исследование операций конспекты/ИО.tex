\RequirePackage{ifluatex}
\let\ifluatex\relax

\documentclass[aps,%
12pt,%
final,%
oneside,
onecolumn,%
musixtex, %
superscriptaddress,%
centertags]{article} %% 
\topmargin=-40pt
\textheight=650pt
\usepackage[english,russian]{babel}
\usepackage[utf8]{inputenc}
%всякие настройки по желанию%
\usepackage[colorlinks=true,linkcolor=black,unicode=true]{hyperref}
\usepackage{euscript}
\usepackage{supertabular}
\usepackage[pdftex]{graphicx}
\usepackage{amsthm,amssymb, amsmath}
\usepackage{textcomp}
\usepackage[noend]{algorithmic}
\usepackage[ruled]{algorithm}
\usepackage{lipsum}
\usepackage{indentfirst}
\usepackage{babel}
\usepackage{pgfplots}
\usepackage{setspace}
\linespread{1.15}
\pgfplotsset{compat=1.9}

\pgfplotsset{model/.style = {blue, samples = 100}}
\pgfplotsset{experiment/.style = {red}}

\selectlanguage{russian}

\theoremstyle{plain}
\binoppenalty=10000
\newtheorem{theorem}{Theorem} %
\setlength{\parindent}{2.4em}
\setlength{\parskip}{0.1em}
\newtheorem{lemma}{Лемма}

\setlength{\parindent}{2.4em}
\setlength{\parskip}{0.1em}
%\renewcommand{\baselinestretch}{2.0}

\usepackage{xcolor}
\usepackage{hyperref}

\begin{document}

\begin{titlepage} 
\begin{center}
% Upper part of the page
%\textbf{\Large САНКТ-ПЕТЕРБУРГСКИЙ ГОСУДАРСТВЕННЫЙ ЭКОНОМИЧЕСКИЙ УНИВЕРСИТЕТ} \\[1.0cm]
%\textbf{\large Кафедра Прикладной Математики и Информатики}\\[3.5cm]
 
% Title
\textbf{}\\[10.0cm]
\textbf{\LARGE Исследование операций}\\[0.5cm]
\textbf{\Large ПМ-1701} \\[0.2cm]

%supervisor
\begin{center} \large
{Преподаватель:} \\[0.5cm]
\textsc {Чернов Виктор Петрович}\\
{viktor\_chernov@mail.ru}\\
\end{center}
% \begin{flushright} \large
%\emph{Рецензент:} \\
%д.ф. - м.н., профессор \textsc{Надеемся Нам Помогут}
%\end{flushright}
%\begin{flushright} \large
%\emph{Заведующий кафедрой:} \\
%д.ф. - м.н., профессор \textsc{Не Обмани Себя}
%\end{flushright}
\vfill 

% Bottom of the page
{\large {Санкт-Петербург}} \par
{\large {2020 г., 6 семестр}}
\end{center} 
\end{titlepage}

% Table of contents
\begin{thebibliography}{3}
\bibitem{Sulsky1994}
Sulsky D., Chen Z., Schreyer H. L.  A particle method for history-dependent materials // Computer Methods in Applied Mechanics and Engineering. --- 1994, V. 118. --- P. 179--196.
\bibitem{LiuLiu}
Liu G. R., Liu M. B. Smoothed particle hydrodynamics: a meshfree particle method. --- Singapore : World Scientific Publishing. --- 2003. --- 449 p.
\end{thebibliography}
\tableofcontents
\newpage
\section{Конспекты лекций}
\subsection{13.02.2020} 

\textbf{Отчет о результатах:} в каких пределах можно менять коэффициенты целевой функции чтобы оптимальынй план не изменился.

Перейдем к листу отчета об устойчивости.

\textbf{Теневая цена} - предельная полезность ресурса, компонент оптимального плана двойственной задачи, частная производная целелвой функции по правой части ограничения - величина, показывает на сколько единиц изменится результат, если изменить правую часть на единицу.

Представим задачу, меняем коэффициенты правой части, получили оптимальное решение $z^{*}$:
$$CX \to max$$
$$ \left\{
\begin{matrix}
AX \leq B \\ 
X \geq 0
\end{matrix}\right. $$
$$ Z^{*} = Z(B) = Z(b_1,b_2,...,b_n)$$
$$ \frac{\partial Z}{\partial b_i} =y_i^*$$
где $y_i^*$ - теневые цены, компоненты оптимального плана.

График предельной полезности является кусочно-линейным.

Отчет о пределах - сомнительная польза: если объем печенья будем равны $0$, то остается один бисквит.

\subsection{20.02.2020}
\subsubsection{Стратегии управления запасами и критерий оптимальности}

Рисуем типичный график зависимости запасов от времени. В начальный момент времени есть какой-то запас и он изменяется с течением времени. Склад является аккумулятором запасов потребителя. На склад, в свою очередь постсупает продукция поставщиков.

В какой-то момент времени запас склада пополняется на некоторую величину $V_1$.
Дефицит может отображаться двумя способами. 
\begin{itemize}
	\item Незадолженный дефицит - спустя какое-то время на склад при нулевом запасе приходит товар
	\item Задолженный дефицит - дефицит уходит в отрицательную область.
\end{itemize}

Последовательность пополнения запасов - результат принятия решений, она возникает тогда, когда потребительская система формирует заказ поставщикам.
$$ \left\{
\begin{matrix}
V_1 & V_2 & \ldots \\
t_1 & t_2 & \ldots
\end{matrix} \right.$$

Данный график носит название \textit{стратегии управления поставками}. Она состоит из отдельных управленческих решений. Какой график поставок лучше, т.е какая стратегия оптимальна?
В этом и состоит оптимизационная задача.

Сущестует три вида затрат:
\begin{itemize}
	\item Затраты связаны с поставками
	\item Затраты связаны с хранением
	\item Затраты связаны с дефицитом
\end{itemize}

Каждая из затрат подразделяется на постоянные и переменные затраты
Постоянные - не зависещее от объема. Затраты, связанные с поставкой, не зависят от объема: затраты на огранизацию.


Критерий оптмимальности: средние затраты в единицу времени были минимальными.

\subsubsection{Простейшие модели управления запасами. Формула Уилсона.}

Простешйая модель обладает тремя свойствами:

\begin{enumerate}
	\item Дефицит не допускается.
	\item Постоянный не меняющийся спрос, $\alpha$ -сколько единиц товара уходит на единицу времени
	\item Отсутствует неопределенность
\end{enumerate}

На графике мы заменяем кривые прямыми, угол наклона будет одинаковым по второму свойству.
Можно предположить, что поставка будет приходить точно в срок, и быть уверенным, что все так и будет.

Оптимальную стратегию следует искать среди графиков следующего вида:

\begin{center}
	\begin{tikzpicture}
		\begin{axis}[xmin=0,xmax = 16, grid = major,scale = 0.9,domain = 0:13]
		\legend{ 
	Начальные данные
	};
	\addplot[color=red,mark=x] coordinates { (0,5) (5,0) (5,3) (8,0) (8,4) (12,0)} ;
		\end{axis}
	\end{tikzpicture}
\end{center}

Обозначим за $a$ - постоянные затраты поставок. Постоянные затраты связанные с хранением мы устраняем из рассмотрения. Переменная составляющая по поставкам - тоже исключается, так как мы на нее не можем влиять - она изменяется от нас не зависяще. $b$ - коэффициент затрат по хранению - затраты по хранению товара на единицу времени. Размерность - количество единиц товара на единицу времени.
Дефицитные поставки все исключаем.

Коэффициент $b$ на графике - единичный квадрат.

Допустим у нас есть два треугольника. Общие затраты равны суммы двух затрат $T = T_1 + T_2$, $Q=\alpha \cdot T$
Тогда средние затраты равны площади этих двух треугольников, то есть:
$$mse =  \frac{2a + b(\frac{1}{2}Q_1T_1+\frac{1}{2}Q_2T_2)}{T}$$

Так как $Q=\alpha \cdot T$, то:
$$mse= \frac{2a + b(\frac{1}{2}\alpha T_1^2+\frac{1}{2} \alpha T_2^2)}{T}$$

Необходимо минимизировать следующее выражение:
$$2a + \frac{1}{2}b\alpha(T_1^2+(T-T_1)^2) \to \min$$

Возьмем производную:
$$f'(T_1) = b\alpha(T_1 - (T-T_1)) = b\alpha (-T + 2T_1)=0$$
$$T_1 = \frac{1}{2}T,T_2 = \frac{1}{2}T$$

Следовательно, оптимальные решения нужно искать среди периодической модели с одинаковыми треугольниками. Теперь задача состоит в том, чтобы найти длину партии $Q$ и $T$ - пероид.

Затраты на одном цикле управления запасами:
$$L_{sum} = a+\frac{1}{2}bQT = a+b\frac{1}{2}\alpha T^2$$

Такие формулы не позволятют сравнивать стратегии, следовательно нужно сравнить средни затраты, поэтому поделим на длину цикла:
$$ L = \frac{a+b\frac{1}{2}\alpha T^2}{T} = \frac{a}{T} + \frac{1}{2} \cdot b \cdot \alpha\cdot  T \to \min$$
$$L'(T) = -\frac{a}{T^2} + \frac{1}{2}b\alpha = 0$$
$$T^* = \sqrt{\frac{2a}{b\alpha}} - \min$$
$$Q^* = \sqrt {\frac{2a\alpha}{b}} - \min$$
$$ L = \frac{a}{\sqrt{\frac{2a}{b\alpha}}} + \frac{1}{2}b\alpha \sqrt{\frac{2a}{b\alpha}} = 
\sqrt{\frac{ab\alpha}{2}} + \sqrt{\frac{ab\alpha}{2}} = \sqrt{2ab\alpha}$$

Данные формулы называются \textit{Формулами Уилсона}.
Если рассмотреть зависимость двух величин $L$ от $T$, то графически мы ищем минимум зеленой прямой на графике:
\begin{center}
	\begin{tikzpicture}
		\begin{axis}[xmin=0,xmax = 5, grid = major,scale = 1.1,domain = 0:5]
	\addplot[color=red] {3/x} ;
	\addlegendentry{затраты по поставкам}
	\addplot[color=blue] {1/2*3*2*x} ;
	\addlegendentry{затраты по хранению}
	\addplot[color=green] {3/x + 1/2*3*2*x} ;
	\addlegendentry{средние затрыты L}
		\end{axis}
	\end{tikzpicture}
\end{center}

Необходимо выбрать прямоугольник заданной площади с минимальным периодом и данный прямоугольник является квадратом.

Философское правило: лучше перебрать, чем недобрать.

\subsubsection{Простейшая модель с допущением незадолженного дефицита.}

\begin{center}
	\begin{tikzpicture}
		\begin{axis}[xmin=0,xmax = 16, grid = major,scale = 0.9,domain = 0:13,title=Незадолженный дефицит]
		\legend{ 
	Начальные данные
	};
	\addplot[color=red,mark=x] coordinates { (0,4) (5,0) (8,0) (8,4) (12,0)} ;
		\end{axis}
	\end{tikzpicture}
\end{center}

Обозначим за $T_1$ недефицитный период $(0;4): T_1$ и $(4,8):T_2$ - период дефицтного периода. $g$ - штраф за отсутствие товара.
$$\alpha,a,b,g, Q = \alpha \cdot T_1$$

$$L = \frac{a+b\frac{1}{2}Q \cdot T_1+g\cdot T_2}{T_1 + T_2} \to \min$$
Лемма о неправильной суммы дробей:
\begin{lemma}
	$\frac{A_1}{B_1} \leq \frac{A_2}{B_2}$
\end{lemma}
\textit{Доказательство:}
$$\frac{A_1}{B_1} \leq \frac{A_1+A_2}{B_1+B_2} \leq \frac{A_2}{B_2}$$
$$A_1B_1 +A_2B_2 \leq A_1B_1 + A_2B_1 $$
$$\frac{A_1}{B_1} \leq \frac{A_2}{B_2} \qquad \blacksquare$$

$\sqrt{2a\alpha b} < g$ - дефифит не выгоден,
$\sqrt{2a\alpha b} > g$ - выгоден дефицит.

\subsubsection{Простешная модель с задолженным дефицитом}
$$ X = \alpha T_1, S = \alpha T_2, \alpha,a,b,g$$
$$ Q = \alpha T $$

$S $ - задолженный дефицит.
$$L = \frac{a+bT_1X\frac{1}{2} + g T_2 S \frac{1}{2}}{T_1 + T_2} \to \min$$
$$L = \frac{a+bT_1^2\alpha \frac{1}{2} + g T_2^2 \alpha \frac{1}{2}}{T_1 + T_2} \to \min$$

Приравниваем к нулю производные уравнений и решаем систему.

$$T_2 = \frac{b}{g}T_1$$
$$T_1^* = \sqrt{\frac{2a}{b\alpha \cdot (1+\frac{b}{g})}}$$
$$ T_2^* = \frac{b}{g}  \sqrt{\frac{2a}{b\alpha \cdot (1+\frac{b}{g})}} =  \sqrt{\frac{2agb^2}{b\alpha \cdot (g+b) g^2}} =\sqrt{\frac{2ab}{\alpha \cdot (g+b) g}} $$
В пределе:
$$T_1^* \to \sqrt{\frac{2a}{b\alpha }}$$
$$T_2^* \to 0$$
$$X*=\alpha T_1 = \alpha  \sqrt{\frac{2a}{b\alpha \cdot (1+\frac{b}{g})}} $$

Размер дефицита:
$$ S^* = \alpha T_2 = \alpha \sqrt{\frac{2ab}{\alpha \cdot (g+b) g}}$$

То есть при оптимальном случае, размер дефицита стремится к нулю, а $X \to Q$.

\subsubsection{Модель с растянутой поставкой и задолженным дефицитом.}

В тот момент, когда приходит поставка, запас увеличивается по какой-то линейной функции с каким-то угловым коффициентом. Разгрузка товара проходит с какой-то скоростью $\beta$. $\alpha$ - скорость уменьшения запаса (интенсивность спроса - объем разгружаемого товара в единицу времени). $\beta - \alpha$ - угол наклона прямой разгрузки поставки.

$\alpha$ - угловой коэффициент ($\tg \alpha$)
В модели с дефицитом запасы уходят в минус и со скоростью $\beta - \alpha$ повышаются.

% вставить рисунок.

$T_1'$ - поставка есть. $T_1''$ - поставки нет. $T_1$ - запас есть. $T_2$ - дефицит.
Максимальный размер запаса $X$, максимальный размер дефицита $S$. 

$$ X = (\beta-\alpha) \cdot T_1' = \alpha T_1''$$
$$ S = (\beta-\alpha) \cdot T_2' = \alpha T_2''$$

a - постоянные затраты не зависящие от объема.

Определим средние затраты.
$$L = \frac{a+b\frac{1}{2}T_1X + g \frac{1}{2}T_2 S}{T} = \frac{a+b\frac{1}{2}T_1X + g \frac{1}{2}T_2 S}{T_1 + T_2} \to \min$$

Должны минимизировать относительно $T_1,T_2,X,S$.

$$T_1' = \frac{x}{\beta - \alpha}, \quad T_1 = \frac{X}{\alpha} \Rightarrow T_1 = \frac{(\alpha+\beta-\alpha)X}{\alpha(\beta-\alpha)}$$

$$ X = \frac{\alpha(\beta-\alpha)}{\beta}T_1 = \lambda T_1 $$
$$ S = \frac{\alpha(\beta-\alpha)}{\beta}T_2 = \lambda T_2 $$

Подставим:
$$ L = \frac{a+b\frac{1}{2}\lambda T_1^2 + g \frac{1}{2}T_2^2 \lambda}{T_1 + T_2} \to \min $$

Возьмем частные производные:

$$ \frac{\partial L}{\partial T_1} = \frac{b\lambda T_1(T_1+T_2) - (a + \frac{1}{2}b\lambda T_1^2 + \frac{1}{2}g\lambda T_2^2)}{(T_1+T_2)^2} = 0$$
$$ \frac{\partial L}{\partial T_1} = \frac{g\lambda T_1(T_1+T_2) - (a + \frac{1}{2}b\lambda T_1^2 + \frac{1}{2}g\lambda T_2^2)}{(T_1+T_2)^2} = 0$$
$$ (T_1+T_2) \lambda (bT_1 - gT_2) =0 \Rightarrow T_2 = \frac{b}{g}T_1$$
$$ b\lambda T_1(T_1+\frac{b}{g}T_1) - (a + \frac{1}{2}b\lambda T_1^2 + \frac{1}{2}g\lambda (\frac{b}{g}T_1)^2) = 0$$
$$ \frac{1}{2}b\lambda T_1^2 (1+\frac{b}{g}) = a$$

Найдем оптимальные значения:
$$T_1^* = \sqrt \frac{2a}{b\lambda (1+\frac{b}{g})}, \quad \lambda = \frac{\alpha(\beta-\alpha)}{\beta} $$
$$T_2^* = \frac{b}{g} \sqrt \frac{2a}{b\lambda (1+\frac{b}{g})}, \quad \lambda = \frac{\alpha(\beta-\alpha)}{\beta} $$
$$X^* = \sqrt \frac{2a\lambda}{b (1+\frac{b}{g})}, \quad \lambda = \frac{\alpha(\beta-\alpha)}{\beta} $$
$$S^* = \frac{b}{g} \sqrt \frac{2a\lambda}{b (1+\frac{b}{g})}, \quad \lambda = \frac{\alpha(\beta-\alpha)}{\beta} $$

Если $\frac{b}{g} \to \min , T_2^* = 0, S^* = 0$, то получится бездефицитная модель.
Выразим $\lambda$: 
$$ \lambda = \alpha (1-\frac{\beta}{\alpha})$$
Если $\alpha \to 0$, то $\lambda \to \alpha $. И прямая становится все более вертикальной - разбираемся с растяжкой.

Домашнее задание - разобрать модель с растянутой поставкой и незадолженным дефицитом.

\subsection{Теория массового обслуживания}

\subsubsection{Структура систем массового обслуживания}

Есть некоторый поток входящих требований. Детерменированные потоки - потоки, которые подчиняются некому расписанию. Регулярный поток - поток с постоянным интервалом между соседними элементами. Перед тем как попасть на очередь, используется \textit{накопитель}. Накопитель может быть ограниченным или неограниченным. 

После этого  - парралельно работающие устройства - узлы обслуживания. Сам процесс обсулживания - случайный процесс.Длительность обслуживания может быть разной. Посе прохождения обсужливания получается выходящий поток требований.

\subsubsection{Три свойства потоков требования}

Изучение потока требований нацелено на получение важнейших его характеристик. Характеристики вероятностного потока, естественно, являются вероятностными. К ним относятся, например, вероятности поступления того или иного числа требований на заданном отрезке времени, среднее число требований, поступающих за данное время, вероятностное распределение длин временных интервалов между соседними требованиями и т.д. Оказывается, первая из названных характеристик является фундаментальной: зная ее, можно определить остальные. 

Мы введем для нее специальное обозначение: характеристика потока требования на промежутке.

$V_k(t_0,t) $ -  вероятность возникновения $k$-требований из рассматриваемого потока на промежутке времени, начинающийся в $t_0$ и имеет длину $t$.

$V_{\geq k} (t_0,t) $ -  вероятность возникновения не менее $k$-требований из рассматриваемого потока на промежутке времени, начинающийся в $t_0$ и имеет длину $t$, $V_{\leq k} (t_0,t) $ - не более $k$.

При этом $V_0(t_0,t)$ становится вероятностью отсутствия требований на нашем отрезке времени.

$V_{\geq 1} (t_0,t)$ - возникновение хотя бы одного требования.

\textbf{Свойства потока требования:}

\textit{1. Стационарность потока.} 

Поток называется стационарным, если его базовая характеристика $V_k(t_0,t) $ не зависит от $t_0$, то есть не зависит от положения отрезка на оси времени (вероятность не зависит от положения на оси).
$$ V_k(t_0,t) = V_k(t_0',t) \eqno(1.3)$$

\textit{2. Ординарность потока.} 

Поток называется ординарным, если требования возникают по одному.

Рассмотрим вероятность возникновения на каком-то промежутке времени более двух требований.$V_{\geq 2} (t_0,t)$. Устремим конец к началу, тогда данная вероятность будет стремиться к нулю. Для того чтобы уловить ординарность необходимо, чтобы данная вероятность стремилась быстро к нулю.

Еще одно эквивалентное определение можно дать через бесконечно малую величину - величина, стремящаяся к нулю быстрее, чем $t$:
$$ V_{\geq 2} (t_0, t) = o(t) \eqno (1.5)$$

Если поток является стационарным, то условие ординарности упрощается и приобретает вид:
$$\lim_{t \to 0}  \frac{V_{\geq 2} (t_0,t)}{t} = 0 \eqno(1.6)$$
$$ V_{\geq 2} (t) = o(t) \eqno (1.7)$$

\textit{3. Отсутсвует последействие}

У потока отсутствует последействие, если его вероятностные характеристики, связанные с разными промежутками времени являются независимыми.

\textbf{Задание 1.2:}

Выведем формулу (1.8)

\textit{Доказательство:}

Возьмем на оси времени два промежутка $t \text{ и } \tau$. 

Определим вероятноть того, что за время $t+\tau$ событие наступит ровно $k$ раз. Это может осуществиться $k+1$ различными способами, а именно:

\begin{itemize}
	\item за промежуток времени длительности $t$ произойдет $k$ событий, а за время $(t+\tau)$ - ни одного
	\item за промежуток времени длительности $t$ произойдет $k-1$ событий, а за время $(t+\tau)$ - 1
	\item за промежуток времени длительности $t$ произойдет $k-2$ событий, а за время $(t+\tau)$ - 2 
	\item $\ldots$
	\item за $k+1$ промежуток времени длительности $t$ не наступит ни одного события, а за время $(t+\tau)$  - k событий.
\end{itemize}

Воспользуемся формулой полной вероятности, а именно найдем вероятность наступление $k$ событий равна:

$$V_k(t_0,t+\tau) = \sum_{m=0}^{k} V_m(t_0,t) \cdot V_{k-m} (t_0+t,\tau)\quad \blacksquare \eqno (1.8) $$

$$\sum_{k=0}^{\infty} V_k(t_0,t) = 1$$

- знание истории не дает уточнить что-то в будущем.

Если поток удовлетворяет всем трем свойствам, то такой поток является \textit{Пуассоновским}.

\textbf{Задание 1.1: Примеры потоков:}


\begin{enumerate}
	\item стационарный + ординарный + отсутствие последствий: падение капли из не до конца завинченного крана.
	\item стационарный + ординарный + последствия: проходящая баржа под разведенными мостами ночью
	\item стационарный + не ординарный + последствия: машины, въезжающих на Володарский мост 
	\item стационарный + не ординарный + отсутствие последствий: поток пассажиров входящих в метро
	\item не стационарный + ординарный + последствия: появление поезда из туннеля в метро
	\item не стационарный + ординарный + отсутствие последствий: выход из квартиры человека
	\item не стационарный + не ординарный + отсутствие последствий: поток уборки станций в одно и то же время.
	\item не стационарный + не ординарный + последствия: появление вагонов из туннеля в метро
\end{enumerate}

\subsubsection{Параметр и интенсивность потока}


Конспекты с лекций:

\textit{Опр:} Параметром потока называется предел вероятности возникновения хотя бы одного требования:
$$\lim_{t \to 0}  \frac{V_{\geq 1} (t_0,t)}{t} = \lambda (t_0)$$
$$\lim_{t \to 0}  \frac{V_{\geq 1} (t)}{t} = \lambda$$

$\lambda$ - параметр потока.

\textit{Опр:} рассмотрим математическое ожидание числа требования на промежутке времени $\mathbb{E}(t_0,t)$:
$$\mathbb{E}(t_0,t) = \sum_{k=0}^{\infty} k \cdot V_k(t_0,t) =  \sum_{k=1}^{\infty} k \cdot V_k(t_0,t) $$.

Будем рассматривать среднее число требования на коротких промежутках времени:

$$ \lim_{t \to 0}  \frac{\mathbb{E}(t_0,t)}{t} = \mu (t_0)$$

$\mu(t_0)$ - мгновенная интенсивность потока.

$$ \lim_{t \to 0}  \frac{\mathbb{E}(t)}{t} = \mu$$

$\mu$ - число, интенсивность потока $\blacksquare$

Введем две важные характеристики потоков: параметр и интенсивность.

Пусть дан стационарный поток. Его параметром называется предел (если он существует для рассматриваемого потока):
$$\lambda =\lim_{t \to 0}  \frac{1 - V_{0} (t)}{t} = \lim_{t \to 0}  \frac{V_{\geq 1} (t)}{t} \eqno (2.1)$$

Параметр обозначается буквой $\lambda$. Из (2.1) следует, что:
$$1 - V_{0} (t) = V_{\geq 1} (t) = \lambda t + o(t) \eqno (2.2)$$

Параметр показывает скорость сходимости к 0 вероятности поступления хотя бы одного требования на отрезке t при стремлении к 0 длины отрезка. Очевидно, что параметр не может быть отрицательным.

Если параметр существует и конечен, то используя (2.2) получаем:
$$V_{\geq 1} (0) = \lim_{t \to 0}  V_{\geq 1} (t) = 0 \eqno (2.3)$$
$$V_0(0) = \lim_{t \to 0}  V_0 (t) = \lim_{t \to 0} (1-V_{\geq 1}(t)) = 1 \eqno (2.4)$$

то есть вероятность поступления хотя бы одного требования в точке (в момент времени, на отрезке времени длины 0) равна 0, а вероятность отсутствия требований в точке равна 1. Это обстоятельство, конечно, не противоречит тому, что в некоторые моменты времени требования поступают; оно связано с бесконечностью множества моментов времени.

Интенсивностью стационарного потока называется среднее число требований, поступающих из потока за единицу времени. Интенсивность обозначается буквой $\mu$. Таким образом:

$$\mu = \lim_{t \to 0} \frac{\mathbb{E}(t)}{t}$$
$$\mathbb{E}(t) =\sum_{k=0}^{\infty} k \cdot V_k(t) = \sum_{k=1}^{\infty} k \cdot V_k(t) \eqno (2.5)$$

Интенсивность потока, очевидно, не может быть отрицательной. Если поток не предполагается стационарным, то значение параметра может меняться во времени. 

Значением параметра в момент $t_0$ (мгновенным значением параметра) называется предел:
$$\lambda (t_0) = \lim_{t \to 0}  \frac{1 - V_{0} (t_0,t)}{t} = \lim_{t \to 0}  \frac{V_{\geq 1} (t_0,t)}{t} \eqno (2.6)$$

Аналогично может менять свое значение и интенсивность. Значением интенсивности в момент $t_0$ (мгновенной интенсивностью) называется предел
$$ \mu (t_0) = \lim_{t \to 0}  \frac{\mathbb{E}(t_0,t)}{t} \eqno (2.7)$$

где математическое ожидание числа требования на промежутке времени $\mathbb{E}(t_0,t)$:
$$\mathbb{E}(t_0,t) = \sum_{k=0}^{\infty} k \cdot V_k(t_0,t) =  \sum_{k=1}^{\infty} k \cdot V_k(t_0,t) \eqno (2.8)$$.

В стационарном случае значение $\lambda(t_0), \mu(t_0)$ постоянны:
$$\lambda(t_0) = \lambda, \quad \mu(t_0) = \mu$$


\textbf{Утв:} Мгновенные параметры и интенсивность связаны следующим соотношением: $$\mu(t_0) \geq \lambda(t_0) \eqno(2.9)$$

\textit{Доказательство:}
$$\mathbb{E}(t_0,t) = \sum_{k=0}^{\infty} k \cdot V_k(t_0,t) =  \sum_{k=1}^{\infty} k \cdot V_k(t_0,t) \geq \sum_{k=1}^{\infty} V_k(t_0,t) = V_{\geq 1}(t_0,t)$$
$$\mathbb{E}(t_0,t) \geq  V_{\geq 1}(t_0,t)\eqno (2.10)$$
$$\mu (t_0) = \lim_{t \to 0}  \frac{\mathbb{E}(t_0,t)}{t} \geq  \lim_{t \to 0}  \frac{V_{\geq 1} (t_0,t)}{t} = \lambda(t_0) \Rightarrow \mu (t_0) \geq \lambda(t_0) \quad \blacksquare$$

\textbf{Задание 2.1:}

\textbf{Утв:} для стационарных потоков выполняется $$\mu \geq \lambda \eqno(2.11)$$

\textit{Доказательство:}
$$ \mu =\lim_{t \to 0} \frac{\sum_{k=0}^{\infty} k \cdot V_k(t)}{t} =\lim_{t \to 0} \frac{ \sum_{k=1}^{\infty} k \cdot V_k(t)}{t} \geq \lim_{t \to 0} \frac{ \sum_{k=1}^{\infty} V_k(t)}{t} =\lim_{t \to 0} \frac{ V_{\geq 1}(t)}{t} = \lambda$$
$$\mu \geq \lambda \quad \blacksquare$$ 

У потоков, моделирующих реальные процессы поступления требований, параметр (то есть предел (2.1) или (2.6)) обычно существует; в дальнейшем мы будем изучать только такие потоки.

Исходя из этого, мы можем теперь дать другую формулировку ординарности стационарных потоков, эквивалентную (1.6). 
$$\lim_{t \to 0}  \frac{V_{\geq 2} (t)}{t} = 0 \eqno(1.6)$$

\textbf{Утв:} поток является ординарным тогда и только тогда, когда:
$$ \lim_{t \to 0} \frac{V_{\geq 2}(t)}{V_{1}(t)} = 0 \eqno (2.15)$$

\textit{Доказательство:}

Верно, что $$V_1(t) = V_{\geq 1}(t) - V_{\geq 2}(t) \eqno (2.16)$$

Откуда получаем:
$$ \lim_{t \to 0} \frac {V_1(t) }{t} =\lim_{t \to 0} \frac {V_{\geq 1}(t) }{t} - \lim_{t \to 0} \frac {V_{\geq 2}(t) }{t} = \lambda - \lim_{t \to 0} \frac {V_{\geq 2}(t) }{t} \eqno (2.17) $$

Пусть поток ординарен, то есть выполнено (1.6). Тогда из (2.17) следует:
$$ \lim_{t \to 0} \frac {V_1(t) }{t} = \lambda \eqno (2.18)$$

$$\lim_{t \to 0} \frac{V_{\geq 2}(t)}{V_1(t)} = \lim_{t \to 0} \frac{V_{\geq 2}(t)}{t} \cdot \lim_{t \to 0} \frac{t}{V_1(t)} = 0 \eqno (2.19)$$

Достаточность тоже доказывается. $\blacksquare$

\textbf{Утв:} для стационарных потоков с конечной интенсивностью из условия $\mu = \lambda$ следует условие ординарности.


\textbf{Задание 2.2:}

\textbf{Утв:} 
поток называется ординарным тогда и только тогда, когда
$$\lim_{t \to 0} \frac{V_{\geq 2}(t)}{V_{1}(t)} = 0 \sim \lim_{t \to 0} \frac{V_{\geq 2}(t)}{V_{\geq 1}(t)} = 0 $$

\textit{Доказательство:}

Необходимость: 

$$ \lim_{t \to 0} \frac{V_{\geq 2}(t)}{V_{\geq 1}(t)} = \lim_{t \to 0} \frac{V_{\geq 2}(t)}{t} \cdot \lim_{t \to 0} \frac{t}{V_{\geq 1}(t)} = 0$$

Достаточность:

Пусть поток удовлетворяет условию:
$$ \lim_{t \to 0} \frac{V_{\geq 2}(t)}{V_{\geq 1}(t)} = 0$$

Тогда:
$$\lim_{t \to 0} \frac{V_{\geq 2}(t)}{V_{1}(t)} = 0 \qquad \blacksquare$$

\textbf{Утв:} 
$$\lim_{t \to 0} \frac{V_{\geq 2}(t)}{V_{1}(t)} = 0 \sim \lim_{t \to 0} \frac{V_{1}(t)}{V_{\geq 1}(t)} = 1 $$

\textit{Доказательство:}
$$\lim_{t \to 0} \frac{V_{1}(t)}{V_{\geq 1}(t)} = \lim_{t \to 0} \frac{V_{1}(t)}{t} \cdot \lim_{t \to 0} \frac{t}{V_{\geq 1}(t)} = \lambda \frac{1}{\lambda} = 1$$

В обратную сторону аналогично $\blacksquare$.

\subsubsection{Определение пуассоновского потока и вычисление вероятности V0}
\subsubsection{Вывод формул для вероятностей Vk элементарным методом}
\subsubsection{Свойства вероятностей Vk пуассоновского потока}

Вероятность $V_k(t)$ обладают следующими свойствами:

1. У каждой вероятности есть единственная точка максимум и сама точка максимума находится на линии предыдщей вероятности

\textit{Доказательство:}
$$f(t) = \frac{(\lambda t)^k}{k!} \cdot e^{-\lambda t} \to \max $$
$$\frac{\partial f(t)}{\partial t} = \frac{\lambda k (\lambda t)^{k-1}}{k!} \cdot e^{-\lambda t} - \lambda e^{-\lambda t} \frac{(\lambda t)^k}{k!} = 0$$
$$e^{-\lambda t} \cdot \frac{(\lambda t)^{k-1}}{(k-1)!} = e^{-\lambda t} \frac{(\lambda t)^k}{k!} $$

Действительно, точка максимума одна и в стационарной точке, являющейся максиумом, совпадает с предыдущей вероятностью. $\blacksquare$

2. Точки максимумов располагаются равномерно 

\textit{Доказательство:}

При $\lambda \neq 0, k \leq 1$ и, так как $e^{-\lambda t} \neq 0 \Rightarrow$:
$$ t^{k-1} \cdot ({\frac{\lambda}{k}t - 1})=0$$
$$t=0, t=\frac{k}{\lambda}$$

Следовательно, получили равномерную последовательность с шагом $t=\frac{k}{\lambda}$. .

3. Значение точек максимумов убывают с увеличением $t$ $\blacksquare$

\textit{Доказательство:}
$$f(\frac{k}{\lambda}) = \frac{k^k}{k!}e^{-k} = \frac{1}{k!}\cdot \left (\frac{k}{e} \right )^k \sim \frac{1}{\sqrt {2\pi k}}$$

Следовательность максимумов стремится к нулю, что и требовалось доказаать в данном свойстве $\blacksquare$
\newpage
4. Параметр $\lambda $ равен интенсивности $\mu (2.5)$.

\textit{Доказательство:}
$$\mathbb{E}(t) =\sum_{k=0}^{\infty} k \cdot V_k(t) = \sum_{k=1}^{\infty} k \cdot V_k(t) (2.5) = \sum_{k=1}^{\infty} k \cdot \frac{(\lambda t)^k}{k!} \cdot e^{-\lambda t} =e^{-\lambda t}  \sum_{k=1}^{\infty} \frac{(\lambda t)^k}{(k-1)!} =$$
$$=e^{-\lambda t} \lambda t \sum_{k=1}^{\infty} \frac{(\lambda t)^{k-1}}{(k-1)!} = e^{-\lambda t} \cdot \lambda t \cdot e^{\lambda t} = \lambda t \quad \blacksquare$$

\end{document}