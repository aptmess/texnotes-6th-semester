\RequirePackage{ifluatex}
\let\ifluatex\relax

\documentclass[aps,%
12pt,%
final,%
oneside,
onecolumn,%
musixtex, %
superscriptaddress,%
centertags]{article} %% 
\topmargin=-40pt
\textheight=650pt
\usepackage[english,russian]{babel}
\usepackage[utf8]{inputenc}
%всякие настройки по желанию%
\usepackage[colorlinks=true,linkcolor=black,unicode=true]{hyperref}
\usepackage{euscript}
\usepackage{supertabular}
\usepackage[pdftex]{graphicx}
\usepackage{amsthm,amssymb, amsmath}
\usepackage{textcomp}
\usepackage[noend]{algorithmic}
\usepackage[ruled]{algorithm}
\usepackage{lipsum}
\usepackage{indentfirst}
\usepackage{babel}
\usepackage{pgfplots}
\usepackage{setspace}
\linespread{1.15}
\pgfplotsset{compat=1.9}

\pgfplotsset{model/.style = {blue, samples = 100}}
\pgfplotsset{experiment/.style = {red}}

\selectlanguage{russian}

\setlength{\parindent}{2.4em}
\setlength{\parskip}{0.1em}
%\renewcommand{\baselinestretch}{2.0}

\usepackage{xcolor}
\usepackage{hyperref}

\begin{document}

\begin{titlepage} 
\begin{center}
% Upper part of the page
%\textbf{\Large САНКТ-ПЕТЕРБУРГСКИЙ ГОСУДАРСТВЕННЫЙ ЭКОНОМИЧЕСКИЙ УНИВЕРСИТЕТ} \\[1.0cm]
%\textbf{\large Кафедра Прикладной Математики и Информатики}\\[3.5cm]
 
% Title
\textbf{}\\[10.0cm]
\textbf{\LARGE Исследование операций}\\[0.5cm]
\textbf{\Large ПМ-1701} \\[0.2cm]

%supervisor
\begin{center} \large
{Преподаватель:} \\[0.5cm]
\textsc {Чернов Виктор Петрович}\\
{viktor\_chernov@mail.ru}\\
\end{center}
% \begin{flushright} \large
%\emph{Рецензент:} \\
%д.ф. - м.н., профессор \textsc{Надеемся Нам Помогут}
%\end{flushright}
%\begin{flushright} \large
%\emph{Заведующий кафедрой:} \\
%д.ф. - м.н., профессор \textsc{Не Обмани Себя}
%\end{flushright}
\vfill 

% Bottom of the page
{\large {Санкт-Петербург}} \par
{\large {2020 г., 6 семестр}}
\end{center} 
\end{titlepage}

% Table of contents
\begin{thebibliography}{3}
\bibitem{Sulsky1994}
Sulsky D., Chen Z., Schreyer H. L.  A particle method for history-dependent materials // Computer Methods in Applied Mechanics and Engineering. --- 1994, V. 118. --- P. 179--196.
\bibitem{LiuLiu}
Liu G. R., Liu M. B. Smoothed particle hydrodynamics: a meshfree particle method. --- Singapore : World Scientific Publishing. --- 2003. --- 449 p.
\end{thebibliography}
\tableofcontents
\newpage
\section{Конспекты лекций}
\subsection{13.02.2020} 

\textbf{Отчет о результатах:} в каких пределах можно менять коэффициенты целевой функции чтобы оптимальынй план не изменился.

Перейдем к листу отчета об устойчивости.

\textbf{Теневая цена} - предельная полезность ресурса, компонент оптимального плана двойственной задачи, частная производная целелвой функции по правой части ограничения - величина, показывает на сколько единиц изменится результат, если изменить правую часть на единицу.

Представим задачу, меняем коэффициенты правой части, получили оптимальное решение $z^{*}$:
$$CX \to max$$
$$ \left\{
\begin{matrix}
AX \leq B \\ 
X \geq 0
\end{matrix}\right. $$
$$ Z^{*} = Z(B) = Z(b_1,b_2,...,b_n)$$
$$ \frac{\partial Z}{\partial b_i} =y_i^*$$
где $y_i^*$ - теневые цены, компоненты оптимального плана.

График предельной полезности является кусочно-линейным.

Отчет о пределах - сомнительная польза: если объем печенья будем равны $0$, то остается один бисквит.

\subsection{20.02.2020}
\subsubsection{Стратегии управления запасами и критерий оптимальности}

Рисуем типичный график зависимости запасов от времени. В начальный момент времени есть какой-то запас и он изменяется с течением времени. Склад является аккумулятором запасов потребителя. На склад, в свою очередь постсупает продукция поставщиков.

В какой-то момент времени запас склада пополняется на некоторую величину $V_1$.
Дефицит может отображаться двумя способами. 
\begin{itemize}
	\item Незадолженный дефицит - спустя какое-то время на склад при нулевом запасе приходит товар
	\item Задолженный дефицит - дефицит уходит в отрицательную область.
\end{itemize}

Последовательность пополнения запасов - результат принятия решений, она возникает тогда, когда потребительская система формирует заказ поставщикам.
$$ \left\{
\begin{matrix}
V_1 & V_2 & \ldots \\
t_1 & t_2 & \ldots
\end{matrix} \right.$$

Данный график носит названя стратегии \textit{управления поставками}. Какой график поставок лучше?
В этом и состоит оптимизационная задача.

Сущестует три вида затрат:
\begin{itemize}
	\item Затраты связаны с поставками
	\item Затраты связаны с хранением
	\item Затраты связаны с дефицитом
\end{itemize}

Каждая из затрат подразделяется на постоянные и переменные затраты
Постоянные - не зависещее от объема. Затраты, связанные с поставкой, не зависят от объема: затраты на огранизацию.






\end{document}