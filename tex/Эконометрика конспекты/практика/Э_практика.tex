\RequirePackage{ifluatex}
\let\ifluatex\relax

\documentclass[aps,%
12pt,%
final,%
oneside,
onecolumn,%
musixtex, %
superscriptaddress,%
centertags]{article} %% 
\topmargin=-40pt
\textheight=650pt
\usepackage[english,russian]{babel}
\usepackage[utf8]{inputenc}
%всякие настройки по желанию%
\usepackage[colorlinks=true,linkcolor=black,unicode=true]{hyperref}
\usepackage{euscript}
\usepackage{supertabular}
\usepackage[pdftex]{graphicx}
\usepackage{amsthm,amssymb, amsmath}
\usepackage{textcomp}
\usepackage[noend]{algorithmic}
\usepackage[ruled]{algorithm}
\usepackage{lipsum}
\usepackage{indentfirst}
\usepackage{babel}
\usepackage{pgfplots}
\pgfplotsset{compat=1.9}

\pgfplotsset{model/.style = {blue, samples = 100}}
\pgfplotsset{experiment/.style = {red}}
\selectlanguage{russian}

\setlength{\parindent}{2.4em}
\setlength{\parskip}{0.1em}
%\renewcommand{\baselinestretch}{2.0}

\usepackage{xcolor}
\usepackage{hyperref}
 
 % Цвета для гиперссылок
%\definecolor{linkcolor}{HTML}{799B03} % цвет ссылок
%\definecolor{urlcolor}{HTML}{799B03} % цвет гиперссылок
 
%\hypersetup{pdfstartview=FitH,  linkcolor=linkcolor,urlcolor=urlcolor, colorlinks=true}

\begin{document}

\begin{titlepage} 
\begin{center}
% Upper part of the page
%\textbf{\Large САНКТ-ПЕТЕРБУРГСКИЙ ГОСУДАРСТВЕННЫЙ ЭКОНОМИЧЕСКИЙ УНИВЕРСИТЕТ} \\[1.0cm]
%\textbf{\large Кафедра Прикладной Математики и Информатики}\\[3.5cm]
 
% Title
\textbf{}\\[10.0cm]
\textbf{\LARGE Эконометрика}\\[0.5cm]
\textbf{\Large ПМ-1701} \\[0.1cm]

%supervisor
\begin{center} \large
{Преподаватель:} \\[0.5cm]
\textsc {Курышева Светлана Владимировна }\\
%{viktor\_chernov@mail.ru}\\
\end{center}
% \begin{flushright} \large
%\emph{Рецензент:} \\
%д.ф. - м.н., профессор \textsc{Надеемся Нам Помогут}
%\end{flushright}
%\begin{flushright} \large
%\emph{Заведующий кафедрой:} \\
%д.ф. - м.н., профессор \textsc{Не Обмани Себя}
%\end{flushright}
\vfill 

% Bottom of the page
{\large {Санкт-Петербург}} \par
{\large {2020 г., 6 семестр}}
\end{center} 
\end{titlepage}

% Table of contents
\begin{thebibliography}{3}
\bibitem{eliseeva}
Эконометрика: Учебник/И.И.Елисеева и др.-М.:Проспект, 2009
\bibitem{praktikuma}
Практикум по эконометрике: Учебное пособие/И.И.Елисеева и др.,М.:Финансы и статистика,2006 
\bibitem{dop1}
Эконометрика: Учебник/В. С.Мхитарян и др.-М.:2008
\bibitem{dop2}
Доугерти К. Введение в эконометрику: Учебник. 2-е изд. / Пер. с англ. – М.: ИНФРА – М, 2007
\bibitem{dop3}
Берндт Э. Практика эконометрики: классика и современность. М.,2005
\end{thebibliography}
\tableofcontents
\newpage

\section{21.02.2020}

Дана зависимость спроса от цены:
$$ X = (15,14,12,11,9,10) $$
$$ Y = (5,4,6,10,18,10) $$

\textbf{1.} Необходимо построить поле корреляции и выбрать математическую функцию.

\begin{center}
	\begin{tikzpicture}
		\begin{axis}[xmin = 0, xmax = 20,ymin = 0, ymax = 20, grid = major,scale = 1.5,legend pos = north west]
		\legend{ 
	Начальные данные
	};
	\addplot[scatter,only marks] coordinates { (15,5) (14,4) (12,6) (11,10) (9,18) (10,10)} ;
		\end{axis}
	\end{tikzpicture}
\end{center}

По данной информации лучшей аппроксимации является нелинейная регрессия - степенная функция.

\textbf{2.} Найти линейное уравнение, используя МНК.
$$y = a + bx$$
Согласно формуле (7) получаем следующую систему уравнений:
$$ \left\{
\begin{matrix}
53 = 6a + 71b \\
575 = 71a + 867b \\
\end{matrix} \right. $$
Из данной системы уравнений находим значения параметров регрессии $a$ и $b$:
$$ a = 31.8385; b = -1.9441$$

Построим график прямой $$\widehat{Y} = 31.8385 -1.9441X$$
\begin{center}
	\begin{tikzpicture}
		\begin{axis}[xmin=0,xmax=20, grid = major,scale = 1.5,domain = 0:25]
		\legend{ 
	Уравнение регрессии, 
	Начальные данные
	};
	\addplot[dashed, model]{31.8385 -1.9441*x};
	\addplot[scatter,only marks] coordinates { (15,5) (14,4) (12,6) (11,10) (9,18) (10,10)} ;
		\end{axis}
	\end{tikzpicture}
\end{center}

Линейный коэффициент корреляции по формуле (10):
$$ r = -0.87378$$

\textbf{3.} Построить таблицу дисперсионного анализа:

\label{first_table_analiz}
\begin{table}[H]
	\begin{center}
		\begin{tabular}[t]{|c|c|c|c|c|} \hline
		Источник вариации & df & $SS$ & $MS$ & F-критерий\\ \hline
		Регрессия (r) & 1 & 101.417 & 101.417 & 12.9127 \\ \hline
		Остаток (e)& 4 & 31.4161 & 7.85404 & 1 \\ \hline
		Итого (t)& 5 & 132.833 & 26.5667 & x \\ \hline
		\end{tabular}
	\caption{Таблица дисперсионного анализа для примера}
	\end{center}
\end{table}

Найдем табличное значение распределения Фишера-Снедекора при заданном уровне значимости:
$$ F_{1-\alpha}(m,n-1-m) = F_{0.95}(1,4) = 7.71$$

\textbf{4.} Найти линейное уравнение регрессии, используя программу Excel.

\subsection{Hometask}

\textbf{5.} Дать интервальный прогноз спроса, для $x_p=9$

Выражение для \textbf{стандартной ошибки предсказываемого по линии регрессии значения} $\widehat{y}$:
$$ m_{\widehat{y_x}} = \sqrt{MS_E} \sqrt{\frac{1}{n}+\frac{(x_k-\overline{X})^2}{\sum(X-\overline{X})^2}} \eqno (37) $$

Для прогнозируемого значения $\widehat{y}$ доверительный интервал выглядит следующим бразом:
$$ \widehat{y_{x_k}} \pm t_{1-\frac{\alpha}{2}} \cdot m_{\widehat{y_x}} $$
$$ \widehat{y_{x_k}} - t_{1-\frac{\alpha}{2}} \cdot m_{\widehat{y_x}} \leq \widehat{y_{x_k}} \leq \widehat{y_p} + t_{1-\frac{\alpha}{2}} \cdot m_{\widehat{y_x}}  \eqno (38)$$

\textbf{Средняя ошибка прогнозируемого индивидуального значения} составит:
$$ m_y = \sqrt{MS_E} \sqrt{1+\frac{1}{n}+\frac{(x_k-\overline{X})^2}{\sum(X-\overline{X})^2}} \eqno (39)$$

\textbf{Доверительный интервал для $y_p$ }- предсказываемого значения регрессии:
$$ \widehat{y_p} - t_{\alpha}m_y \leq y_p \leq \widehat{y_p} + t_{\alpha}m_y \eqno (40) $$

Вычислим стандартную ошибку предсказываемого по линии регрессии значения $\hat{Y}$ по формуле (37):
$$ m_{\widehat{y_x}} = \sqrt{7.85404} \sqrt{\frac{1}{6} + \frac{(x_k-11.8333)^2}{26.8333}} $$

Подставляя различные значения из выборки $X$ мы можем узнать ошибку предсказываемого значения. Минимальная ошибка будет при подстановке $x_k = \overline{X}=11.8333$:
$$ m_{y_{\overline{X}}} = \sqrt{11.8333} \sqrt{\frac{1}{6}} = 1.14412$$

Построим доверительный интервал для $\widehat{Y}$ при каком-то произвольном значении $x_k$, например $x_k = 9$. Воспользуемся формулой (38).

Сначала вычислим значение линейной регресии в точке $x_k=9$:
$$ \widehat{y_{9}} =  31.8385 -1.9441 \cdot 9 = 14.3416$$

Затем вычислим стандартную ошибку в точке $x_k=9$:
$$ m_{\widehat{y_9}} = \sqrt{7.85404} \sqrt{\frac{1}{6} + \frac{(9-11.8333)^2}{26.8333}} = 1.91278$$

Теперь можно и построить доверительный интервал для уровня значимости $\alpha=0.05$:
$$ \widehat{y_{9}} - t_{0.975}\cdot m_{\widehat{y_9}} \leq \widehat{y_{9}} \leq \widehat{y_{9}} + t_{0.975}\cdot m_{\widehat{y_9}}$$
$$9.0309 \leq \widehat{y_{9}} \leq 19.6523$$

Средняя ошибка прогнозируемого индивидуального значения:
$$ m_y = \sqrt{MS_E} \sqrt{1+\frac{1}{n}+\frac{(x_k-\overline{X})^2}{\sum(X-\overline{X})^2}} \eqno  = \sqrt{7.85404} \sqrt{1+\frac{1}{6}+\frac{(9-11.8333)^2}{26.8333}} = 3.39304 $$

\textbf{Доверительный интервал для $y_p$ }- предсказываемого значения регрессии: 
$$ 4.92101 \leq \widehat{y_{9}} \leq 23.7622$$

\textbf{6.} Используя Excel найти уравнение регрессии по степенной функции.

\begin{center} 1. Степенная функция \end{center}

Модель:
$$ y = a \cdot x^b \cdot \varepsilon $$

Логарифмируем обе части равенства (линеаризация):
$$ \ln y =\ln a + b\ln x + \ln \varepsilon $$

Замена переменных:
$$ \ln y = z, \alpha_1 = \ln a, t = \ln x, \varepsilon_1 = \ln \varepsilon $$

Линейный вид:
$$ z =\alpha_1 + b \cdot t + \varepsilon_1 $$

В нашем случае нам нужно прологарифмировать наши ряды $x$ и $y$, найти коэффициенты $a$ и $b$ линейной функции и перейти обратно к степенной, сделав замену $a={\rm e}^{\alpha_1}$

$$ X = (15,14,12,11,9,10) $$
$$x_{new} = \ln X = (2.70805, 2.63906, 2.48491, 2.3979, 2.19722, 2.30259)$$
$$ Y = (5,4,6,10,18,10) $$
$$ y_{new} = \ln Y = (1.60944, 1.38629, 1.79176, 2.30259, 2.89037, 2.30259)$$
Находим коэффициенты методом МНК:

$$ \alpha_1 = 8.58854, b = -2.66456 $$

Исходная степенная модель имеет вид:

$$ y = {\rm e}^{\alpha_1}x^b = 5369.78 \cdot x^{-2.66456} $$

Построим график степенной функции $$y= 5369.78 \cdot x^{-2.66456}$$

\begin{center}
	\begin{tikzpicture}
		\begin{axis}[xmin=8,xmax=16, grid = major,scale = 1.5,domain = 8:16]
		\legend{ 
	Уравнение регрессии, 
	Начальные данные
	};
	\addplot[dashed, model]{5369.78*x^-2.66456};
	\addplot[scatter,only marks] coordinates { (15,5) (14,4) (12,6) (11,10) (9,18) (10,10)} ;
		\end{axis}
	\end{tikzpicture}
\end{center}
\end{document}

