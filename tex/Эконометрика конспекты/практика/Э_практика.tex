\RequirePackage{ifluatex}
\let\ifluatex\relax

\documentclass[aps,%
12pt,%
final,%
oneside,
onecolumn,%
musixtex, %
superscriptaddress,%
centertags]{article} %% 
\topmargin=-40pt
\textheight=650pt
\usepackage[english,russian]{babel}
\usepackage[utf8]{inputenc}
%всякие настройки по желанию%
\usepackage[colorlinks=true,linkcolor=black,unicode=true]{hyperref}
\usepackage{euscript}
\usepackage{supertabular}
\usepackage[pdftex]{graphicx}
\usepackage{amsthm,amssymb, amsmath}
\usepackage{textcomp}
\usepackage[noend]{algorithmic}
\usepackage[ruled]{algorithm}
\usepackage{lipsum}
\usepackage{indentfirst}
\usepackage{babel}
\usepackage{pgfplots}
\pgfplotsset{compat=1.9}

\pgfplotsset{model/.style = {blue, samples = 100}}
\pgfplotsset{experiment/.style = {red}}
\selectlanguage{russian}

\setlength{\parindent}{2.4em}
\setlength{\parskip}{0.1em}
%\renewcommand{\baselinestretch}{2.0}

\usepackage{xcolor}
\usepackage{hyperref}
 
 % Цвета для гиперссылок
%\definecolor{linkcolor}{HTML}{799B03} % цвет ссылок
%\definecolor{urlcolor}{HTML}{799B03} % цвет гиперссылок
 
%\hypersetup{pdfstartview=FitH,  linkcolor=linkcolor,urlcolor=urlcolor, colorlinks=true}

\begin{document}

\begin{titlepage} 
\begin{center}
% Upper part of the page
%\textbf{\Large САНКТ-ПЕТЕРБУРГСКИЙ ГОСУДАРСТВЕННЫЙ ЭКОНОМИЧЕСКИЙ УНИВЕРСИТЕТ} \\[1.0cm]
%\textbf{\large Кафедра Прикладной Математики и Информатики}\\[3.5cm]
 
% Title
\textbf{}\\[10.0cm]
\textbf{\LARGE Эконометрика}\\[0.5cm]
\textbf{\Large ПМ-1701} \\[0.1cm]

%supervisor
\begin{center} \large
{Преподаватель:} \\[0.5cm]
\textsc {Курышева Светлана Владимировна }\\
%{viktor\_chernov@mail.ru}\\
\end{center}
% \begin{flushright} \large
%\emph{Рецензент:} \\
%д.ф. - м.н., профессор \textsc{Надеемся Нам Помогут}
%\end{flushright}
%\begin{flushright} \large
%\emph{Заведующий кафедрой:} \\
%д.ф. - м.н., профессор \textsc{Не Обмани Себя}
%\end{flushright}
\vfill 

% Bottom of the page
{\large {Санкт-Петербург}} \par
{\large {2020 г., 6 семестр}}
\end{center} 
\end{titlepage}

% Table of contents
\begin{thebibliography}{3}
\bibitem{eliseeva}
Эконометрика: Учебник/И.И.Елисеева и др.-М.:Проспект, 2009
\bibitem{praktikuma}
Практикум по эконометрике: Учебное пособие/И.И.Елисеева и др.,М.:Финансы и статистика,2006 
\bibitem{dop1}
Эконометрика: Учебник/В. С.Мхитарян и др.-М.:2008
\bibitem{dop2}
Доугерти К. Введение в эконометрику: Учебник. 2-е изд. / Пер. с англ. – М.: ИНФРА – М, 2007
\bibitem{dop3}
Берндт Э. Практика эконометрики: классика и современность. М.,2005
\end{thebibliography}
\tableofcontents
\newpage
\section{21.02.2020}

Дана зависимость спроса от цены:
$$ X = (5,4,6,10,18,10) $$
$$ Y = (15,14,12,11,9,10) $$

1.Необходимо построить поле корреляции и выбрать математическую функцию.

\begin{center}
	\begin{tikzpicture}
		\begin{axis}[xmin = 0, xmax = 20,ymin = 0, ymax = 20, grid = major,scale = 1.5,legend pos = north west]
		\legend{ 
	Начальные данные
	};
	\addplot[scatter,only marks] coordinates { (15,5) (14,4) (12,6) (11,10) (9,18) (10,10)} ;
		\end{axis}
	\end{tikzpicture}
\end{center}

По данной информации лучшей аппроксимации является нелинейная регрессия - степенная функция.

2. Найти линейное уравнение, используя МНК.
$$y = a + bx$$
Согласно формуле (7) получаем следующую систему уравнений:
$$ \left\{
\begin{matrix}
53 = 6a + 71b \\
575 = 71a + 867b \\
\end{matrix} \right. $$
Из данной системы уравнений находим значения параметров регрессии $a$ и $b$:
$$ a = 31.8385; b = -1.9441$$

Построим график прямой $$\widehat{Y} = 31.8385 + -1.9441X$$
\begin{center}
	\begin{tikzpicture}
		\begin{axis}[xmin=0,xmax=20, grid = major,scale = 1.5,domain = 0:25]
		\legend{ 
	Уравнение регрессии, 
	Начальные данные
	};
	\addplot[dashed, model]{31.8385 -1.9441*x};
	\addplot[scatter,only marks] coordinates { (15,5) (14,4) (12,6) (11,10) (9,18) (10,10)} ;
		\end{axis}
	\end{tikzpicture}
\end{center}

Линейный коэффициент корреляции по формуле (10):
$$ r = -0.87378$$

3. Построить таблицу дисперсионного анализа:

\label{first_table_analiz}
\begin{table}[H]
	\begin{center}
		\begin{tabular}[t]{|c|c|c|c|c|} \hline
		Источник вариации & df & $SS$ & $MS$ & F-критерий\\ \hline
		Регрессия & 1 & 101.417 & 101.417 & 12.9127 \\ \hline
		Остаток & 4 & 31.4161 & 7.85404 & 1 \\ \hline
		Итого & 5 & 132.833 & 26.5667 & x \\ \hline
		\end{tabular}
	\caption{Таблица дисперсионного анализа для примера}
	\end{center}
\end{table}

Найдем табличное значение распределения Фишера-Снедекора при заданном уровне значимости:
$$ F_{1-\alpha}(n-1-m,n) = F_{0.95}(1,4) = 7.71$$

4. Найти линейное уравнение регрессии, используя программу Excel.
5. Дать интервальный прогноз спроса, предполагая, что $x_p=9$
6. Используя Excel найти уравнение регрессии по степенной функции.
\end{document}

