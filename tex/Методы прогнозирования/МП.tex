\RequirePackage{ifluatex}
\let\ifluatex\relax

\documentclass[aps,%
12pt,%
final,%
oneside,
onecolumn,%
musixtex, %
superscriptaddress,%
centertags]{article} %% 
\topmargin=-40pt
\textheight=650pt
\usepackage[english,russian]{babel}
\usepackage[utf8]{inputenc}
%всякие настройки по желанию%
\usepackage[colorlinks=true,linkcolor=black,unicode=true]{hyperref}
\usepackage{euscript}
\usepackage{supertabular}
\usepackage[pdftex]{graphicx}
\usepackage{amsthm,amssymb, amsmath}
\usepackage{textcomp}
\usepackage[noend]{algorithmic}
\usepackage[ruled]{algorithm}
\usepackage{lipsum}
\usepackage{indentfirst}
\usepackage{babel}
\usepackage{pgfplots}
\usepackage{setspace}
\linespread{1.15}
\pgfplotsset{compat=1.9}

\pgfplotsset{model/.style = {blue, samples = 100}}
\pgfplotsset{experiment/.style = {red}}

\selectlanguage{russian}

\setlength{\parindent}{2.4em}
\setlength{\parskip}{0.1em}
%\renewcommand{\baselinestretch}{2.0}

\usepackage{xcolor}
\usepackage{hyperref}
 
 % Цвета для гиперссылок
%\definecolor{linkcolor}{HTML}{799B03} % цвет ссылок
%\definecolor{urlcolor}{HTML}{799B03} % цвет гиперссылок
 
%\hypersetup{pdfstartview=FitH,  linkcolor=linkcolor,urlcolor=urlcolor, colorlinks=true}

\begin{document}

\begin{titlepage} 
\begin{center}
% Upper part of the page
%\textbf{\Large САНКТ-ПЕТЕРБУРГСКИЙ ГОСУДАРСТВЕННЫЙ ЭКОНОМИЧЕСКИЙ УНИВЕРСИТЕТ} \\[1.0cm]
%\textbf{\large Кафедра Прикладной Математики и Информатики}\\[3.5cm]
 
% Title
\textbf{}\\[10.0cm]
\textbf{\LARGE Методы прогнозирования}\\[0.5cm]
\textbf{\Large ПМ-1701} \\[0.1cm]

%supervisor
\begin{center} \large
{Преподаватель:} \\[0.5cm]
\textsc {Ивахненко Дарья Александровна}\\
{viktor\_chernov@mail.ru}\\
\end{center}
% \begin{flushright} \large
%\emph{Рецензент:} \\
%д.ф. - м.н., профессор \textsc{Надеемся Нам Помогут}
%\end{flushright}
%\begin{flushright} \large
%\emph{Заведующий кафедрой:} \\
%д.ф. - м.н., профессор \textsc{Не Обмани Себя}
%\end{flushright}
\vfill 

% Bottom of the page
{\large {Санкт-Петербург}} \par
{\large {2020 г., 6 семестр}}
\end{center} 
\end{titlepage}

% Table of contents
\begin{thebibliography}{3}
\bibitem{rob}
Теория игр
\end{thebibliography}
\tableofcontents
\newpage

\section{11.02.2020}
\subsection{Введение}
Будем исследовать временные ряды и варианты его прогнозирования. В конце семестра перейдем к машинному обучению.

Мы рассматривали введение в файле и семейство преобразований Бокса-Кокса:

$$ y^{(\lambda)} = 
\left\{
\begin{matrix}
\frac{y^{\lambda}-1}{\lambda}, \lambda \neq 0 \\
\ln y, \lambda = 0
\end{matrix} \right. \eqno $$
$$ \lim_{\lambda \to 0}{\frac{y^{\lambda}-1}{\lambda}} = \lim_{\lambda \to 0}{\frac{\exp (\lambda \ln y)-1}{\lambda}} = \lim_{\lambda \to 0}{\frac{1+\lambda \ln y+O((\lambda \ln y)^2) - 1}{\lambda}} = \ln y $$

$$y^{\lambda} = \frac{y^\lambda -1}{\lambda} $$


$$
\frac{\partial y^{\lambda} }{\partial \lambda} = \frac{\lambda y^{\lambda - 1}}{\lambda} = y^{\lambda-1} $$

Параметр $\lambda$ можно найти с помощью метода максимального правдоподобия:
$$ L = \prod_{i=1}^{T} {\frac{1}{\sqrt{2\pi}\cdot \sigma}} \cdot \exp \left ( {-\frac{({y_i^{(\lambda)} - \overline{y}^{(\lambda)})^2}}{\sigma^2}}\right) \cdot J(\lambda,y) \to \max $$

Прологорифмируем:
$$ \ln L = \sum_{i=1}^{T} \left ( \ln \frac{1}{\sqrt{2\pi}\cdot \sigma}  - \frac{({y_i^{(\lambda)} - \overline{y}^{(\lambda)})^2} }{2\sigma^2} + \ln J(\lambda,y) \right )  $$

Воспользуемся оценкой выборочной дисперсии:
$$ \hat{\sigma}^2 = \sum_{i=1}^{T} \frac{({y_i^{(\lambda)} - \overline{y}^{(\lambda)}})^2}{T} $$
$$ \ln \frac{1}{\sqrt{2\pi \sigma^2}} = \ln \frac{1}{\sqrt{2\pi}} + \ln \frac{1}{\sqrt {\sigma^2}} = \ln \frac{1}{\sqrt{2\pi}} - \ln \frac{({y_i^{(\lambda)} - \overline{y}^{(\lambda)}})^2}{T}$$ 

Тогда:
$$ \ln L = \sum_{i=1}^{T} \ln \frac{1}{\sqrt{2\pi}} - \ln \frac{({y_i^{(\lambda)} - \overline{y}^{(\lambda)}})^2}{T} - \frac{T}{2} + \log \prod_{i=1}^{T} y_{i}^{\lambda - 1} = $$ 
$$ = \sum_{i=1}^{T} \ln \frac{1}{\sqrt{2\pi}} - \ln \frac{({y_i^{(\lambda)} - \overline{y}^{(\lambda)}})^2}{T} - \frac{T}{2} + (\lambda - 1) \sum_{i=1} \ln y_{i}$$

Взяв производную по параметру $\lambda$ и найдя решение, получим:

$$ y^{(\lambda)} = 
\left\{
\begin{matrix}
\frac{{y+\lambda_2}^{\lambda}-1}{\lambda_1}, \lambda_1 \neq 0 \\
\ln (y+\lambda_2), \lambda_1 = 0
\end{matrix} \right. \eqno $$

\end{document}
